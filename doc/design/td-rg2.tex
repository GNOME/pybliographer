
\chapter{Pybliographer Concepts}
\label{cha:rgconc}

This chapter summarises some basic concepts that you need to
understand before you can use \Pyb.
It briefly describes:
\begin{itemize}
\item What bibliographic items are and how they are described
\item What basic data types are used 
\item How the data is stored and processed
\item How \Pyb\ can help you storing and editing bibliographic
  resources and using them in you work
\end{itemize}

% \chapter{Working with Pybliographer}
% \label{cha:rg2}

% \newcommand{\Pyb}{Pybliographer{}}

% When you work with \Pyb, you will do so most often through one of the
% following windows:

% \begin{itemize}
% \item The Search Window
% \item The Folder Window
% \item The List Window
% \item The Detail Window
% \end{itemize}

% Often you will start with the Search Window. Enter a search term and
% start a search.  (See \ref{fig:searchx1}) The same window allows you to
% query another (possibly remote) database, and to formulate more
% complicated queries.

% Alternatively, you can open a predefined collection of records, a
% \textit{folder} or \textit{list}.  The \textit{Folder Window} allows
% you to select from all the folders and lists available. It can also be
% shown permanently as a left sidebar. Click on the small triangle to
% the left of each entry to open or close a folder in this display. Click
% on the name to display its contents in the list window.

% A list is simply a folder that is found under the label Lists at the
% beginnign of the folder display; together with the entries Marked,
% \dots

% Whatever your way, you will sooner or later see a \textit{List
%   Window}. It shows your current \textit{selection}. Usually records
% are ordered by name of first author and are displayed in a condensed
% format. Those are options that are easily set and changed; it is
% possible to tie thses settings to a particular list; think os a
% shopping list for you rnext visit to the library, that you might like
% always sorted by location.
  
% To inspect an item it is often sufficient to point with the mouse to
% it, and have a tooltip like preview window appear. It allows at the
% same time to concisely inspect  the entry and also to preview it in a
% typical format (both can be configured, of course).  Alternatively,
% you may open the deatil view (inspector) for an item. It presents you
% with a lot of information in a notebook widget. 







%%% Local Variables: 
%%% mode: latex
%%% TeX-master: "td-td2"
%%% End: 
