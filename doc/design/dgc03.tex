
\chapter{Architecture Overview}
\label{cha:archiview}

This chapter describes the architecture and components.


\section{Application Core}
\label{sec:archicore}

\Pyb\ provides access to \textbf{Databases} of bibiographical
informations and \textbf{Documents} (\textbf{Resources}). It allows
\textbf{Users} to store and retrieve records in an \textbf{internal
  Database}, including the ability to import the results of
\textbf{Queries} against external databases.  

It organises the stored records in \textbf{Lists} and
\textbf{Folders}, maintains \textbf{Indexes} on various attributes,
as well as \textbf{Authority files}, such as for \textbf{Persons},
\textbf{Subject headings}, \textbf{Classifications}, and other
\textbf{Descriptors}. 

Items in the database can be equipped with \textbf{Annotations}; these
include \textbf{Actions} that are requested (and assorted status),
\textbf{Abstracts} or \textbf{Notes} reflecting the contents, and also
\textbf{Quotations} taken from the source document, structured
\textbf{Case annotations} for further processing, and also
\textbf{Proxy references} to external documents which might serve the
same purposes. 

\textbf{Holdings} are recorded, which include references to electronic
copies as well as shelf numbers or other location information, both at
the user's site or any other institution, anywhere.

The bibliographical \textbf{Description} allows for
\textbf{Extensions} to deal with e.g., special materials.  For
multi-level description a \textbf{Hierarchy} of items is created. 

Items may be assigned a \textbf{Security Label}, to restrict their
dissemination. 

Items are formatted according to \textbf{Formats} that are user
selectable and -definable. 


\section{Architecure Baseline}
\label{sec:archibase}


\section{Implementation Plan}
\label{sec:archiplan}




%%% Local Variables: 
%%% mode: latex
%%% TeX-master: "DG"
%%% End: 
