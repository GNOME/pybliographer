% $Id$

\section{Interfacing to  External Ressources}
\label{sec:external}





\subsection{Connecting to external databases}
\label{sec:extconn}

Pybliographer uses \textit{connections} to external ressources for the
following purposes:
\begin{itemize}
\item To look up a catalogue (for immediate consumption, so to say),
  i.e., as a client program.
\item To obtain, as a result of a query, additional data for import.
\item To obtain additional documents that are referred to from the
  data base, i.e., via an \textit{URL} or an \textit{Eprints} link.
\item To perform operations on the remote system, i.e., to request
  books or copies by means of an \textit{OPAC}. 
\item \Think To cooperate with other Pybliographers, sharing and
  updating data.

\end{itemize}

% Query processing


% * Configuration

%   A list of available connections (databases) is built during start-up
%   (from pyblio.xml) Note that this precludes specialised parsing. 

%   It would be nice to have a configuration dialogue. 

% * Query dialogue

%   Presents a dropdown list of such servers (but allows local search as
%   well). 

%   Determines and loads extension module. (Using 'type' atttribute.)

%   Builds Query object.

% * Extension module

%   Reads and sets parameters (stored during start-up). 

%   Customises query:
%   - marks fields available for search
%   - add-on options for query dialogue (in preference to:)
%   - expert search dialogue

%   Builds request (object) -> Query.request.
  
% * Query requestor

%   Sends request and receives result.  Error processing.
   
%   Determines and loads Importer module. 
%   Builds an Import object. Feeds result data to import module.



\subsection{Writing Import Filters}
\label{sec:importfilter}

Import filters are used not only when a \textit{file} is imported, in
a narrow sense, but also when opening a file, and, perhaps more
importantly, when processing the results of a \textit{query}.

An \textit{Import Module} provides the needed support. Typically, there
is one such module for every major data format, say MARC or Bib\TeX.

A major part of such a module is a Reader class, which subclasses (is
derived from, and specialises) the class \texttt{ImportReader} in the
module \texttt{Pyblio.Import}. 

In fact, most Reader modules do not derive directly from
\texttt{InportReader}, but from an intermediary, as
\texttt{TaggedReader} for data structured as a sequence of tagged
fields, like in the MARC format, \texttt{TextReader} for data that
comes as `unstructured' text, like taken from the references of an
article, \texttt{XmlReader} for XML formatted text.

In this way, we have a \textit{framework} for import modules, one that
makes writing or modifying an import reader much more simple. This is
important, because requirements and interface continue to evolve. 

The format specific modules are not the end of the story, yet. For
one, a format like MARC is in fact so vast that it will probably never
be completely implemented and used, neither by a library, nor by
Pybliographer. It is, however, to be expected that one time or the
other, someone would be better off if he could make use of that exotic
feature, and would be happy to add support. In addition, one may
encounter a situation in which the standard processing is inadequate
and better be modified.  Thus it is useful to be able to have an easy
way of adapting to one's very special needs, be it in terms of adding
or of overriding standard behaviour, i.e., methods.

There is yet another point to be made. A framework is also a way of
sharing the implementation effort. That means that adding code to the
common base is more attractive  and this in turn makes the use of
Pybliographer more attractive. So  everone profits. 



\subsubsection{Framework Services}




The base and intermediate classes, the framework, provide the
following services and features: 
\begin{description}
\item[Input] The source data is accepted by various means: files,
  iterators, in-core objects, and made available in a standardised way
  to the processing. 
\item[User interfacing] The user interface, always requiring a big
  programming effort, is structured by the base classes (togerther, of
  course, with the Gui Component), thus sharing a lot of work.
\item[Data handling] In the past, the handling of the input data was
  almost completely left to the format specific code, resulting in an
  enormous duplication of code and often poor exploitation of the
  source data. With the new design, most of the semantic data handling
  will be done by functions in specialised modules, sharply
  distinguishing between the parsing (syntactic) and database aspects
  of the task.
\item[Duplication checking] Like the user interface, the duplication
  check is an example of a pragramming task, that is not easily
  undertaken within the confines of one particular application, but
  only if it can be done wholesale. 
\item[On- and offline correction] It is highly desirable to be able to
  correct imported data in a systematic way, the framework will provide
  for that, both during inital processing, dubbed on-line, as well as
  at a later time.
\item[Customisation] There are ways to adapt the import processing
  which can be done easily in a generic way.  It is, e.g., simple to
  exclude all data from certain categories (tags) from further
  processing and therefore \texttt{TaggedReader} provides a parameter
  to specify such tags -- it can even be done interactively from an
  options dialogue. No programming needed.
\item[Data management aspects] The data imported will be kept
  logically identified as long as it is desired, without the need to
  keep separate files for it. That remedies a major problem with the
  work organisation in the past and it is necessary in any way for a
  data base oriented Pybliographer.
\end{description}



\subsubsection{Class hierarchy}

The fundamental class is \texttt{ImportReader} in Module
\texttt{Import.py}. It implements an \texttt{Iterator} interface,
viz.\ \texttt{first()} and  \texttt{next()} methods. 

\begin{verbatim}
class ImportReader (Iterator.Iterator):

    """Base class for all import reader classes.

    Support for input methods, encodings, database and GUI connections, 
    """

### The following need no change as a rule:
    def first (self):
        return self.next()

    def next (self, entry=None, data=None):
        """Process the next entry from the input source.
        entry and data argument are provided for easy testing."""

        e = entry or self.next_entry()
        x = data or self.read_next()
        if self.options.has_key('preserve_input'):
            e.lines = x
        self.entry = e
        return self.parse(x)

\end{verbatim}

Subclasses will need to provide, among others, implementations of
\texttt{read\_next} and \texttt{parse}.

A typical implementation of \texttt{parse} from \texttt{TaggedReader}
follows: 
\begin{verbatim}
    def parse (x):

        self.begin_record(x)
       
## it may be convenient, if the read_next routine already assembles
## continuation lines. Let's assume this

        for i in x:
            tag = i[0:self.tagcol]
            data = i[tagcol:]

            if discardrx and discardrx.match(tag):
                continue
                      
            if _cache.has_key(tag):
                _cache[tag] (self, tag, data)
            else:
                methname = 'do_'+str(tag)

                if hasattr(self, methname):
                    method = getattr(self, methname)
                else :
                    method = self.do_tag
                    
                method(self, tag, data)
                _cache[tag] = method

        self.end_record(x)
        return self.entry

\end{verbatim}

So in this case, for a tag `XYZ', a subclass implemented method
`do\_XYZ (self, tag, data)' would be called, if it exists. If not,
`do\_tag' would be called and as a rule, distinguish according to the
tag, as directed by parameters of the class, whether to ignore, or to
store (in a generic way) the data. 


\subsubsection{Class parameters}


\paragraph{Control objects}

\begin{description}
\item[control=None] specifies the associated control object.
\end{description}


\paragraph{Input} Three possibilities are provided for. The data
parameter is particularly useful for testing. \textit{One and only one
must be selected from the following}

\begin{description}
\item[file=\textit{filename}] if input is from a file,
\item[data=\textit{data object}] if input is from an incore object,
\item[iter=\textit{iterator}] if an iterator is specified.
\end{description}


% * Import module

%   Parses result data. Builds result set -> Import.set

%   Special processing:

%   - On-/Off-line correction.
%   - Checking for duplicates.
%   - Checking against author, journals, etc. database.
%   - Flexible handling of input data (store it, ignore it, ...)
  
%   Marks (and displays) the import set as a whole for reference.

% * Control objects

%   Saved in configuration file.
  
%   Static: Connection
%   Dynamic: Query, Import 

  

% $Id$



\subsection{Writing Export Filters}
\label{sec:exportfilter}


\subsection{Pybliographer as Web Server }
\label{sec:pybweb}


\subsection{Requesting remote operations}
\label{sec:remotereq}


\subsection{Requesting documents}
\label{sec:docrequest}



%%% Local Variables: 
%%% mode: latex
%%% TeX-master: "todo"
%%% End: 
