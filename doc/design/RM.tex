\documentclass[12pt,a4paper,DIV8]{scrbook}
\usepackage{pyblio,graphicx,xr}
% add font styles  like: ,galliard,gillsans, if you like
\hyphenation{data-base}
%\rcsInfo $Id$
\externaldocument[DG.]{DG}
\externaldocument[UG.]{UG}
\begin{document}
\frontmatter
\title{Pybliographer Design Handbook}
\author{Fr�d�ric Gobry and Peter Schulte-Stracke}
\maketitle{}\pagestyle{headings}

\newcommand{\txglossbuchstb}[1]{\medskip}

\makeatletter

\newcommand{\txglossentry}[2][]{%
\noindent\textbf{#2\ifx#1\@empty\else\hfill [#1]\fi}\nopagebreak}

\newenvironment{usecaselist}{\begin{itemize}}{\end{itemize}}

\newcommand{\listitem}[3]{\item \textbf{(#1)} {\itshape #2} #3}

\newenvironment{usecase}
{\begin{description}}{\end{description}}

\newcommand{\ucitem}[3]{\textbf{\sffamily #1}\quad
  #2\dotfill\pageref{sec:uc#1}}

\makeatother


\bibliographystyle{acm}
% \begin{abstract}
%    A collection of diagrams and tables detailing the analysis and
%    design of Pybliographer version 2.
% \end{abstract}

\tableofcontents
\reversemarginpar
\mainmatter


\chapter{Use cases}
\label{cha:usecases}

\input{tx-use.ltx}


\chapter{Glossary}
\label{cha:td9gloss}


\input{tx-gloss.lst}


\include{zzuc-all}




\appendix
\cleardoublepage
\backmatter
\bibliography{pyblio}

\end{document}
