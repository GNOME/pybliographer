
\chapter{Preparing to Use Pybliographer}
\label{cha:rgprep}

This chapter discusses how to install and configure \Pyb.


It  describes how to do the following:

\begin{enumerate*}
\item Install \Pyb.
\item Set up the required database structure
  \begin{itemize}
  \item Determine which databases to use
  \item Define the directory structure or access path
  \item Define additional data that you might need
  \end{itemize}
\item Set up \Pyb\ for easier operation
\item Load needed data
\item Write local processing routines
\end{enumerate*}

Ir also discusses how much detail you will need to record, and which
standards you should consider.




\section{Installing Pybliographer}
\label{sec:rginstall}

\textbf{Note:} Before using the information in this section, you
should check the latest \texttt{Readme} file from the distribution for
any changes and additions.

As a rule, \Pyb\ installs like any other Linux program under control
of your package manager. Only in exceptional cases you will have to
consider manual installation. 

On a Debian system you use the package name \texttt{pybliographer}.So
you only need to enter
\begin{verbatim}
sudo apt-get install pybliographer
\end{verbatim}

Note that the libdb3 package will be installed, too. If you want to use
\Pyb\ together with a relational database, you must install it in
addtion to \Pyb, for more information on  using a relational
database see below. 

The following table gives the dependencies of \Pyb. [Is this a good
idea, after all?] \dots

Intructions for manually installing \Pyb.\dots


\subsection{Using a Relational Database with Pybliographer}
\label{sec:rginstdb}

When yo use \Pyb\ with a RDMS, you are in fact using a server process
in conjunction with \Pyb, which is independantly installed, configured
and started. As a rule, this server runs under a separate user
identity. Please, consult the RDMS documentation for more detailed
instructions. 

\textbf{Note:} You may want to check the considerations below in this
chapter before continuing with initialising the database.


\begin{enumerate*}
\item assure that the RDMS server is up and running, and that you have
  any necessary authorisations and informations for its use.
\item Determine the connection to the RBDS 
\item Determine the user id 
\item Start \Pyb. Open the \textsf{Application\slash Preferences} dialogue.
\item \textit{or ?} Open the \textsf{Application\slash Database} window and
  then the \textsf{Database/New} dialogue.
\item Enter the connection data, and tick the \textsf{Default
    Database} Check box.
\item \Pyb\ checks for an  existing infrastructure in the database, as
  it finds none, it alerts you that the database must be initialised.
\item Click the \textsf{OK} button on the message box.
\item The database is initialised.
\end{enumerate*}

\textbf{Note:} You may want to check the considerations in this
chapter before initialising  the database. 

 
\section{Before Using Pybliographer}
\label{sec:rginstprep}

\begin{dnote}
\item Give a task oriented overview here --
\item Refer to chapers one and two for introduction, 
\item refer to Part II for details
\end{dnote}



\section{Customisation: Codes, Lists, and Schemes}
\label{sec:rginstcodes}

\Pyb\ makes use of various lists and data bases of so called
\textit{authority data}.  See section \ref{sec:introdesc} and chapter
\ref{cha:bibl} for a discusion of authority data and the need for it.

In this section we do not address the great authority files for
persons, corporate names and the like, but the smaller  and lesser
ones that deal with codes for languages, genres, realtors et., as
follows.





\section{Customisation: Workforms, Data Entry}
\label{sec:rginstinput}

With \Pyb\ comes a standard list of \textit{work forms} that guide you
through the data entry for new records, and that are also used to
display records according to their needs. 



\section{Customisation: Import Filters}
\label{sec:rginstimport}

\Pyb\ comes with a standard set of import filters.  See
\ref{cha:uimport} for a list and discussion.   This list will suffice
in most cases for the purpose of occasional import and even when
migrating from another system. 

When the exporting system has more than one export option available,
that can be  imported by \Pyb, the results may not be equal; so you
may want to try some tests before settling for one method.

The filters provided are written so as to allow easy customisation,
extending or modifying exiting behaviour,  by subclassing them. You
may profit from this facility in the following situations:
\begin{itemize*}
\item You want to use a new format, that is a variation of an existing
  format (have a look at the examples in chapter \ref{cha:uimport}).
\item You want to use an extension of an existing format; you may want
  to make use of the additional information.
\item You want to make use of an existing format, but change the
  processing.
\item You want to make use of an existing format, and extend the
  treatment of the import data.  One such situation is given when the
  standard filter refrains from using some information because it is
  difficult to map it to the database.  You may be in able to improve
  on this situation, e.g., because you know the context.
\end{itemize*}



\section{Customisation: Output Formats}
\label{sec:rginstformat}


\section{Customisation: External Processors}
\label{sec:rginstexternal}



%%% Local Variables: 
%%% mode: latex
%%% TeX-master: "UG"
%%% End: 
