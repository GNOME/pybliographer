\section{Organising and Manipulating References }
\label{sec:basicwork}

One aspect of Pybliographer is its use to \textit{organise references}
-- that lays the accent onto the varieties of uses that these
references could be put to, and the concomitant variety of possible
organisation we have to cope with. This de-emphasises the
bibliographical description, as a rule, but stresses our ability to
annotate and link.



As a rule, our data is kept in one database, organised into
\textit{folders} (compare Biblioscape \ref{biblioscapefolder}). 

A variety of \textit{Indices} are kept  to allow fast access. q


%%% Local Variables: 
%%% mode: latex
%%% TeX-master: "todo"
%%% End: 
