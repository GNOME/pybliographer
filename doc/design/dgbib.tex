
\chapter{The Bibliographic Component}
\label{cha:bibco}

\begin{description}
\item[Purpose] Description of resources; with accent upon the
  classical bibliographical data (types). -- Contains the core of the
  application domain objects.
\item[Modules] Biblio\dots
\item[External Dependencies] Depends on GUI for display interaction
\item[Internal Dependencies] Depends on Storage for persistence, on
  Common Control for a little configuration and interaction support,
  on Writer for formatting.
\item[Initialisation] Primary DB provides needed configuration and
  control data. Some bootstrapping and fallback mechanism desirable.
\end{description}



\section{Description}
\label{sec:bibcodesc}

\begin{dnote}
  \item It is to be determined, how much of the application domain is
  to be catered for by this package-component. 
\item Demarkation against Storage and Writer seems clear to me, 
\item Candidates for related packages are IMHO Topics, Annotations,
  Holdings 
\end{dnote}

The bibliographical description lies at the heart of \Pyb, together
with the features for annotation and organising of references.

This package provides:
\begin{itemize*}
\item description up to recent ISBD developments, including
  multi-level description and material specific description
\item extensible efficient indexing
\item customisable, assisted  data entry (spellchecking)
\item extensible structured description (case analysis)
\item annotation, integration of external documents w.r.t. searching
\item task-oriented (writing) support 
\item maintenance of associations between bibliographical objects
\item maintenance of classifications, etc.; effient use thereof
\item (some of the above should be expanded upon, in particular the
  following question should be answered: what is the specific
  \textit{bibliographical} content, and what is just generic?
  Example: versions of a document, translations, reviews: how are the
  related?)
\item extensions for collections, archival resources
\end{itemize*}

%%% Local Variables: 
%%% mode: latex
%%% TeX-master: "DG"
%%% End: 
