\setcounter{secnumdepth}{3}
\cleardoublepage


\section{Introduction and Generalities}
\label{sec:intro}


\subsection{An overview of user requirements}
\label{sec:requ}

As a bibliographic database manager, \textit{Pybliographer} places
itself at the intersection of various expectations, traditions, and
requirements, each of which developped originally independent, and
often in ignorance of the other, and which are still shaping the field
according to their own peculiarities, although they are getting into
closer contact recently.

It should prove useful to start with a view of these ancestral
developments:

\begin{itemize}
\item The oldest stems from the librarian's duty to keep inventories
  of his collection in form of \textit{catalogues} -- and the
  associated application software.
\item Very early the scholarly (and technical) document preparation
  has been supported with programs to augment document sources with
  correctly formatted references (\textit{Refer} and \textit{BibTeX}
  are two similar programs for the major document processing
  environments).
\item These programs concerning themselves, however, only with this
  last step, other tools developed over time to help with orgenising
  and searching the databases these programs require. Such
  \textit{reference managers} also aided in the odious task of
  actually creating (i.e., typing) the needed entries.
\item With the growth of Internet usage, it became possible and more
  or less usual to access databases of bibliographical interest
  directly. Both the nature of the data (format, extent, correctness,
  completeness) and of the means of access (protocols, navigation,
  access restrictions, formatting)  vary enormously. Nevertheless, it
  has become desirable to easily integrate the results of these
  searches into one database -- adding a new complexity to the tasks.
\item In addition, more and more the exchange of documents needs to be
  integrated, too. 
\item The World Wide Web may also serve as an imperfect but easily usable
  and ubiquitous way of providing \textit{access} to a database.
\item Finally, the traditional library catalogues have also added
  support for additional administrative functions (both on the
  acqusitions as on the lending side), many of which are quite
  natural extensions of their primary function.
\end{itemize}

In the following some of these strands are more fully discussed.


\subsubsection{Library Catalogues}

Originally inventories for the use of the librarians only, since about
one century, the library catalogues have become accessible to the
public and set the standard of bibliographical description. 

At about the same time began the development of rules for the
bibliographical description,\footnote{The \textit{Preu�ische
    Instruktionen} and the \textit{ALA Rules} were among the first.
  Their successors are the \textit{RAK} and the AACR2, both are in
  turn influenced (the RAK more so) by the international
  standardisation efforts of the \textit{ISBD}.} to enable the sharing
of catalogue entries between libraries\footnote{Either by one library
  providing the catalugue cards f or the majority of the libraries (in
  the US), or by building union catalogues for the purpose of sharing
  the holdings, too (the inter-library lending service in Germany).}
and to enable a catalogue to be build and used \textit{compatibly}
without annoyance by many people.

In 1876 Charles A.~\textsc{Cutter} thus explained the aim of the
catalogue:\footnote{as quoted in \cite[p.196]{hal98}}
\begin{itemize}
\item To enable a person to find a book of which either
  \begin{itemize}
  \item the author is known.
  \item the title is known.
  \item the subject is known.
  \end{itemize}
\item To show what the library has 
  \begin{itemize}
  \item by a given author.
  \item in a given subject.
  \item in a given kind of literature.
  \end{itemize}
\item To assist in a choice of a book 
  \begin{itemize}
  \item as to its edition (bibliographically).
  \item as to its character (litarary or topical).
  \end{itemize}
\end{itemize}

These are still important principles, today perhaps even more than
most of the time. 


\subsubsection{BibTeX}
 


\subsubsection{Reference Managers}


\subsubsection{Database Access}







\subsubsection{Other lists of requirements}
\label{sec:otherreq}



The following is a list of user requirements as given in a talk in the
technical university of Chemnitz\footnote{\url{%
    http://archiv.tu-chemnitz.de/pub/2001/0013/data/anforderungen.htm}} 
\begin{itemize}

\item Flexible acces via various categories, viz. author, title,
  publication year, journal, and combinations thereof.

\item Import and export in various formats, viz. BibTeX, Medline,
  Refer, RFC1807, [MARC, MAB, ...]

\item Supports working groups, allows combinations of databases.

\item Allows to merge data from various sources, and to normalise it
  (e.g., with respect to doffering ways to abbeviate a journal title ...)

\item Adapts to the various formatting requiremants of different
  journals, easy development of new styles.

\item Extensible by local fields, to hold notes, order information,
  annotations, excerpts, summaries, etc.

\item Simple use during document preparation: allows to search the
  references from the editor/wordprocessor (by author, title, e.g.),
  thus eschews the use of labels -- the latter are to be used and
  provided only internally. \seet{wpintegration}

\item Allow confidential data in shared databases (e.g., annotations
  of theses).

\item N.B.: These requiremants transcend BibTeX's abililties; thus it
  should be used only as an export format.
\end{itemize}

Other points that were mentioned on the mailing list:
\begin{itemize}
\item Pybliographer starts too slow; with larger databases one has no
  indication of progress, and must wait for a long time. 
\item Pybliographer holds the whole databse in memory, and uses a lot
  of it! (For comparision: Allegro claims about 600 B (of secondary
  storage) per entry.)

\end{itemize}



\subsubsection{The opportunity for a flexible bibliographic format.}

The following excerpts from a recent googling\cite{mch96} I'd like to
add: The opportunity for a flexible bibliographic format -- Designing
a bibliographic format for highly varied levels of effort:

  \begin{quote}
    It should be possible to create a bibliographic format which
    permits people to use exactly the level of detail they wish,
    rather than requiring or preventing less or more.  This would
    reduce the current undesirable proliferation in bibliographic
    formats.
  \end{quote}


\paragraph{The problem:}
People create bibliographic records with enormously varying levels of
effort.  Further, the distribution of effort across different aspects
of bibliographic information also varies.  Current bibliographic
formats tend to be useful only over a relatively narrow range.  They
establish minimums by requirements on information and its form.  They
establish maximums by limitations in the expressiveness of the format.
They do not gracefully handle attempts to get by with less work, or to
add more work to further refine the record.  This is approximated by
optional fields, but the format and meaning of a fields content is
usually inflexible.  Thus there is an undesirable proliferation of
formats, a new one created whenever someone wants a new tradeoff in
ease vs precision.  The new format is generally as inflexible as those
which preceded it, and the problem thus continues. [\dots]



\paragraph{The opportunity:}
Create a bibliographic format which can gracefully handle a wide range
of cataloging intensity.  One which could handle the complete spectrum
from Bib/Refer to MARC.  And do it in a way which permits work to be
spent only and exactly where one wishes.  Such a format would be a
`lingua franca', as any other bibliographic format could be converted
into it without losing, or requiring additional, information.[\dots]



\subsection{Highlights of Pybliographer Release 2}
\label{sec:highr2}

\begin{itemize}
\item The Index(-view) has been reworked, it now consumes less ressources,
  starts and redraws much faster.
\item ...
\end{itemize}


%%% Local Variables: 
%%% mode: latex
%%% TeX-master: "todo"
%%% End: 
