\setcounter{secnumdepth}{3}
\cleardoublepage


\section{Introduction and Generalities}
\label{sec:intro}


\subsection{An overview of user requirements}
\label{sec:requ}

As a bibliographic database manager, \textit{Pybliographer} places
itself at the intersection of various expectations, traditions, and
requirements, each of which developped originally independent, and
often in ignorance of the other, and which are still shaping the field
according to their own peculiarities, although they are getting in
closer contact recently.

It should prove useful to start with a view of these ancestral
developments:

\begin{itemize}
\item The oldest stems from the librarian's duty to keep inventories
  of his collection in form of \textit{catalogues} -- and the
  associated application software.
\item Very early the scholarly (and technical) document preparation
  has been supported with programs to augment document sources with
  correctly formatted references (\textit{Refer} and \textit{BibTeX}
  are two similar programs for the major document processing
  environments).
\item These programs concerning themselves, however, only with this
  last step, other tools developed over time to help with orgenising
  and searching the databases these programs require. Such
  \textit{reference managers} also aided in the odious task of
  actually creating (i.e., typing) the needed entries.
\item With the growth of Internet usage, it became possible and more
  or less usual to access databases of bibliographical interest
  directly. Both the nature of the data (format, extent, correctness,
  completeness) and of the means of access (protocols, navigation,
  access restrictions, formatting)  vary enormously. Nevertheless, it
  has become desirable to easily integrate the results of these
  searches into one database -- adding a new complexity to the tasks.
\item In addition, more and more the exchange of documents needs to be
  integrated, too. 
\item The World Wide Web may also serve as an imperfect but easily usable
  and ubiquitous way of providing \textit{access} to a database.
\item Finally, the traditional library catalogues have also added
  support for additional administrative functions (both on the
  acqusitions as on the lending side), many of which are quite
  natural extensions of their primary function.
\end{itemize}

In the following some of these strands are more fully discussed.


\subsubsection{Library Catalogues}



\subsubsection{BibTeX}


\subsubsection{Reference Managers}


\subsubsection{Database Access}







\subsubsection{Other lists of requirements}
\label{sec:otherreq}



The following is a list of user requirements as given in a talk in the
technical university of Chemnitz\footnote{\url{%
    http://archiv.tu-chemnitz.de/pub/2001/0013/data/anforderungen.htm}} 
\begin{itemize}

\item Flexible acces via various categories, viz. author, title,
  publication year, journal, and combinations thereof.

\item Import and export in various formats, viz. BibTeX, Medline,
  Refer, RFC1807, [MARC, MAB, ...]

\item Supports working groups, allows combinations of databases.

\item Allows to merge data from various sources, and to normalise it
  (e.g., with respect to doffering ways to abbeviate a journal title ...)

\item Adapts to the various formatting requiremants of different
  journals, easy development of new styles.

\item Extensible by local fields, to hold notes, order information,
  annotations, excerpts, summaries, etc.

\item Simple use during document preparation: allows to search the
  references from the editor/wordprocessor (by author, title, e.g.),
  thus eschews the use of labels -- the latter are to be used and
  provided only internally. \seet{wpintegration}

\item Allow confidential data in shared databases (e.g., annotations
  of theses).

\item N.B.: These requiremants transcend BibTeX's abililties; thus it
  should be used only as an export format.
\end{itemize}

Other points that were mentioned on the mailing list:
\begin{itemize}
\item Pybliographer starts too slow; with larger databases one has no
  indication of progress, and must wait for a long time. 
\item Pybliographer holds the whole databse in memory, and uses a lot
  of it! (For comparision: Allegro claims about 600 B (of secondary
  storage) per entry.)

\end{itemize}




\subsection{Highlights of Pybliographer Release 2}
\label{sec:highr2}

\begin{itemize}
\item The Index(-view) has been reworked, it now consumes less ressources,
  starts and redraws much faster.
\item ...
\end{itemize}


%%% Local Variables: 
%%% mode: latex
%%% TeX-master: "todo"
%%% End: 
