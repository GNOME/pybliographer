\chapter{Introduction}
\label{cha:intro}


\setcounter{secnumdepth}{3}
%\cleardoublepage

Pybliographer provides Reference manager services to create, load,
modify, list, read, transport, and copy descriptions of bibliographic
resources like books, articles, and electronic documents, including
derivative and private information.  In addition it helps in accessing
external databases and document repositories. 

Today, scholars and writers need efficient, consistent, and less
complex ways to work with bibliographic resources and to maintain
their existing inventory of references.  The trend in bibliography is
to integrate and share data, and to develop applications that
transcend traditional boundaries.

In the past, reference managers had only limited ability to adapt to
the plenitude of working environments and to interact with other
applications.  This has often constrained those with special needs to
work with inadequate programs, and burdened them with work-arounds and
makeshift arrangements. Bibliographic software has on the whole tended
to be idiosyncratic and limited in its outlook, or to be expensive,
complicated and nevertheless constrained to a small subset of the
requirements, as it is, e.g., the case with \textsf{MARC}-based
library software (from the point of an individual writer\slash
scholar, at least).

\Pyb\ establishes a common base for many tasks. It combines essential
services, such as database access, formatting and searching records,
with a common vocabulary and structure.  In making use of 
established practice and standards it avoids many of the problems that
other programs have when it comes to extending and inter-operating.
All of these services are available through a set of interfaces that
are easy to adapt and a fertile foundation for extensions.


Many restrictions inherent in the BibTeX database format used by
earlier versions of this program (and many competing programs) are
removed or relaxed in this version. 

This version continues to support \BibTeX\ databases to assist you in
converting to the new database structure. Subsequent releases,
however, might not support these features, so we strongly recommend
conversion to the exclusive use of the new database structure.

Support for using \BibTeX\ as an import and export format, however, is
planned for the foreseeable future.


\section{The Evolution of Bibliographical Software}
\label{sec:introevo}




As a bibliographic database manager, \textit{Pybliographer} places
itself at the intersection of various expectations, traditions, and
requirements, each of which developed originally independent, and
often in ignorance of the other, and which are still shaping the field
according to their own peculiarities, although they are getting into
closer contact recently. For orientation, it may be useful to give a
short view of some of these ancestral developments. They embody still
important principles, and may serve as benchmarks for our endeavour.
 

\subsection{Reference Managers}
\label{sec:introdocrm}

Very early the scholarly (and technical) document preparation has been
supported with programs to augment document sources with correctly
formatted references, of which \textit{BibTeX} may be regarded as
prime example.  Those early applications are slanted towards the
production of reference lists, almost at the exclusion of other uses:
their storage formats are implicitly defined and easy to extend, and
the building and maintaining of its data base is left to other means.

In spite of (or perhaps because of) these deficiencies these have been
very successful programs. Their restrictions, that follow a well
proved tradition after all, were intended and adequate at the time of
their creation. They have been addressed by the next class of
applications which developed, characteristically, in the highly
competitive marketplace of the \textit{personal computer}.

With the advent of the Personal Computer a new class of users came to
the fore. While the majority of the scientific\slash technical writers
(who have little choice, anyway) stayed with their \TeX\ applications,
the new users were mostly attracted to the perceived simplicity of the
word processors, and eschewed the command line and simple editors of
Unix.  Almost by necessity, the new applications stressed those
aspects of the job, that the older tradition neglected; word processor
users didn't care for the typographical refinements of \TeX, nor the
frugality and intellectual self restriction of the command line.  They
wanted simplicity and ease of use, \dots\ and over the time, they
succeeded.  The graphical user interface comes no longer as an
afterthought, it is at the centre of the development effort (often to
the detriment of the application proper).

The features of \textit{Biblioscape}, a commercial product are given
in appendix \ref{sec:commfeat}.
For us it is important to pay attention to the \textit{organising} of
the references, by providing suitable instruments, e.g., the
\textit{virtual folders} of Biblioscape.


\subsection{Library Catalogues}
\label{sec:introlibs}


It comes to no surprise that in the beginning library catalogues were
nothing but simple \textit{inventories}, and that they and their
counterparts in archives, serve this purpose to this day.  After they
became accessible to the public, though, and now it is getting
interesting, a number of problems arose, that are still important, as
are some of the solutions, too.  At about the same time, required by
the inter-library loan services, among others, rules for the
bibliographical description were developed, culminating in the ISBD,
which set a widely accepted standard.  Both developments never ceased
to influence each other.





% {The legacy of the card catalogue --
%   \textsf{MARC}\protect\footnote{Although \textsf{MARC} stands for Machine
%   Readable Cataloguing, it is used here loosely but legitimately and
%   stands for the whole of classical librarian's cataloguing, of which
%   the actual MARC standard has been, in a way the epitome and
%   culmination of centuries of work. Of course, it has begun to change \dots }}
% \label{sec:the-legacy-card}

% Originally inventories for the use of the librarians only, since about
% one century, the library catalogues have become accessible to the
% public and henceforth set the standard of bibliographical description.

% At about the same time began the development of rules for the
% bibliographical description,\footnote{The \textit{Preu�ische
%     Instruktionen} and the \textit{ALA Rules} were among the first.
%   Their successors are the \textit{RAK} and the AACR2, both are in
%   turn influenced (the RAK more so) by the international
%   standardisation efforts of the \textit{ISBD}.} to enable the sharing
% of catalogue entries between libraries\footnote{Either by one library
%   providing the catalogue cards f or the majority of the libraries (in
%   the US), or by building union catalogues for the purpose of sharing
%   the holdings, too (the inter-library lending service in Germany).}
% and to enable a catalogue to be build and used \textit{compatibly}
% without annoyance by many people (which is a form of sharing, too).

The actual rules used have always been a compromise between precision
and cost. % oscillating between the ideal of capturing the whole title
% page (of course, only the title page\footnote{see Robert Musil's
%   \textit{Mann ohne Eigenschaften} and the talk between General Stumm
%   von Borgwehr and the director of the Royal Library \dots}) and the
% need of coping (manually) with all this. 
Over time, the descriptions tended to grow in size. There are several
reasons for this: (i) the more items exist, the more difficult it is
to distinguish between them, the more details are thus needed, (ii) if
one tries to spread the cost over more parties, one has to accept
information that individually one might not find worthwhile, and
finally, (iii) the need for better pre-selection entails even more
information, as, e.g., abstracts, to be given.\footnote {In the
  classical catalogues one could afford to be concise, as one could
  always go and fetch the copy for actual inspection.} 

% \begin{enumerate*}
% \item the more titles exist, the more it becomes difficult to
%   distinguish between similar ones,
% \item more information is made centrally available, mostly to share the
%   cost of providing it. Such was the case in Germany, where subject
%   headings appeared in the national bibliographical database, thus
%   moving something which has traditionally been the responsibility of
%   the individual institution onwards to the \textit{Deutsche
%     Bibliothek}. To do so, additional categories needed to be defined,
%   rules set up, an authority file maintained, as it was done earlier
%   for the nucleus of the bibliographical description.
% \end{enumerate*}

It should be noted, that traditionally only books have been
catalogued, together with other physical objects, that might land in a
library (\textsf{MARC} being quite comprehensive in its scope of
materials allowed). That reflects the point of view of the librarian,
who is the custodian of these said objects, buys, lends, ranges them.
The contents are not his business.  In that his view differs from the
usual patron's view, as exemplified by the Bib\TeX\ and reference
manager software (see \ref{sec:introdocrm}).  This orientation is
slowly changing under the influence of Internet resources and
integration with patron's software (section
\ref{sec:internet-access}).\footnote{\textsf{MARC} intended, indeed,
  to support so called \textit{analytical} cataloguing as well, if
  only as a secondary task. Even the description of archival resources
  is not, or so one thought, out of its scope. This reflects the facts
  that American libraries hold archival collections to a far greater
  degree than European ones, and that these are far simpler than the
  classical European archival fonds; but remarkably this proved to be
  a failure -- a separate standard (EAD) evolved, this time based on
  XML. Perhaps the reason was the absence of any form of hierarchy in
  \textsf{MARC} databases.}

Perhaps the most important heritage from this development is the idea
of \textit{entry points}: try to establish for everything that you
catalogue well-defined and precise enough attributes and make the
database search-able for them. 



For a card catalogue this implies defining not only under which
\textit{form} an author is filed, but in addition under which author a
book is filed (if there is a choice).  The first class of
normalisations is done via \textit{authority files} the importance of
which, under the conditions of on-line catalogues, has only increased,
because it is more difficult to find entries which differ slightly
\cite{Hal98} while the so called question of the main entry has no
longer the importance it used to have.
 


\subsection{Records Management}
\label{sec:intrormapp}

Of course, the management of current administrative records (records
and document management applications) is out of the scope of
bibliographical software, mostly.  But to a degree, it would be
beneficial to be able to handle one's own documents with more ease and
consistency by adding a records management feature to the reference
manager.
 
If, as it is the case in particular in the United States, archival
resources are integrated into library systems, the idea is not far to
use the same software and the same set of rules for both kinds of
material.  The use of \textsf{MARC} for archival resources failed,
however, not because of any impossibility, but because of the cultural
inability of using the in this case inevitable multi-level description
technique.  In Germany, archivists use database systems for data
entry, in America, word processors -- neither choice is fully
adequate.

Extending the database schema to enable the description of
\textit{administrative records, personal papers,} and
\textit{manuscripts}, and collections thereof, should be taken into
consideration. 


\subsection{The spell of the Internet -- Database \& Document Access}
\label{sec:internet-access}

Although the Internet precedes the personal computer by some
years,\footnote{Of course, it all depends.  Arguably the first
  personal computer was the \textsf{Alto} at \textit{Xerox Parc} from
  the early 1970, while the switch to the \textit{IP/TCP} protocol
  family did not occur but a couple of years later.}  the impact of
the Internet was but slowly felt. We can at this time distinguish the
following issues:
\begin{description*}
\item[Database access] in particular with the intent of adding the
  results at least partially to one's database.
\item[Document access] in particular where the document's \textsf{URL}
  forms part of the bibliographical description.
\item[Electronic documents] continue to challenge the librarian's
  profession. Ephemeral, they were often felt not to warrant the
  expense of traditional cataloguing (out of which considerations the
  \textit{Dublin Core} standard evolved), hierarchical, they stretched
  the card-bound structure of \textsf{MARC} to the limits, ever
  changing, they challenged the concepts of bibliographical unit, of
  work, and edition (see chapter \ref{cha:bibdata}).
\end{description*}



\section{Reference Documents}
\label{sec:introrefdoc}

The following gives a list of articles and other reference sources
that are helpful in understanding the issues.


\subsection{General}
\label{sec:introrefgen}
\begin{description}
\item [Research] Dagmar Knorr examines in her thesis \citep{knorr98}
  practical processes of academic writing with a particular regard
  towards the use of bibliographical information and reference manager
  tools.  While she finds a very inhomogeneous landscape, she is
  correct in my view, in stressing the importance of adequate tools,
  and of activities antecedent to and supportive of actual writing,
  such as note taking, including task related notes, and tracing,
  e.g., of quoted literature. She documents an earlier period (early
  1990s) where often very basic Bib\TeX\ features were not available,
  so this would be even relatively more important today.
  
\item [Reviews] Of the many overviews of existing reference manager
  systems, the following have been particular helpful: \dots
\end{description}


\subsection{Cataloguing Rules and Library Applications}
\label{sec:introrefrule}

\begin{description}
\item[ISBD] The influential international standards, available (mostly) at
  \url {www.ifla.org}
\item[FRBR] An important report at \url{}, for discussion papers see \dots
\item[AACR2R] The Anglo-American rules are explained in Maxwell's
  Handbook \cite{MM97}. See also the \dots web site and the following:
\item[MARC] Good official site is \url{www.loc.gov/...} including a lot
  of discussion papers; very good material also at the \textsf{OCLC}
  site, in particular \dots
\item[RAK] The German rules are explained at \dots
\item[Allegro] is a small library system developed at the Technische
  Universit�t in Braunschweig by Eversberg, its documentation \url{}
  comprises an outstanding comparison and discussion of library
  formats.
\end{description}


\subsection{Records Management and Archives}
\label{sec:introrefrecords}
\begin{description}
\item Management of living (current) administrative records has
  evolved, of course, independent of bibliographic considerations. For
  a recent standard see \cite{DOD:5015.2std}.
\item 
\end{description}




%%% Local Variables: 
%%% mode: latex
%%% TeX-master: "DG"
%%% End: 
