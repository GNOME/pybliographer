\addchap{Preface}
\label{cha:preface}

This document serves as the base for development of the \Pyb\
application.This document is intended to give the \Pyb\ developers both an
introduction to the application area and an reference view of the
application as vision, guidance, and motivation.  Accordingly, it is
divided into three parts:

\begin{description*}
\item [Part 1, Overview] introduces the subject area, and lists major
  sources of reference information. Then it treats design goals,
  requested and desired features and functions; it gives an overview
  of the use cases that are to be considered and an overview of teh
  architecture, and the further development.
\item [Part 2, Design Considerations] discusses major areas and
  problems of \Pyb\'s  design  in order to give the development a
  better foundation. 
\item [Part 3, Component Spacifications] contains the specifications
  for all subsystems and packages.
\end{description*}


\section{Where to Find More Information}
\label{sec:prefmore}

 Further information can be found in:
\begin{itemize*}
\item \UG, which contains explanations of the user interactions, data
  elements, and the application programming interface, and
\item \RM, which contains all
  diagrammatic and textual information grouped according to type.
\end{itemize*}

\noindent

\begin{dnote}[Comments:]
\item Comments and corrections are always welcome.
  Please address your message to:
  \url{pybliographer-general@lists.sourceforge.net}.
\end{dnote}


\subsection{Internet Sources}
\label{sec:prefinet}
The following resources are available through the Internet:

\begin{itemize}
\item \textbf{\Pyb\ home page}

  You can vivit the \Pyb\ home page on the World Wode Web using this
  address:
\cbstart
 % \url{http://www.gnome.org/pybliographer}.
   \url{http://pybliographer.sf.net/}. 
% by Fredgo email of Fri, 25 Jul 2003 11:10:49 +0200

\cbend

\item \textbf{\Pyb\ discussion list}

  You can also participate in the discussion on the \Pyb\ general
  mailing list operated through Source Forge. See the following
  information page:\\
  \url{http://lists.sourceforge.net/mailman/listinfo/pybliographer-general}

  To subscribe send a message to:

  \url{pybliographer-general-request@lists.sourceforge.ne} with a
  subject line set to \texttt{Subject: subscribe}.

  To post a question or response on the \Pyb\ general
  mailing list, send it to:

  \url{pybliographer-general@lists.sourceforge.net}

  Include an appropiate \texttt{Subject:} line.

  See the list's archive at the following URL:\\
  \url
{http://sourceforge.net/mailarchive/forum.php?forum=pybliographer-general}.  

\item \textbf{Download area}

  You can get source code and additional information through
  \textbf{anonymous ftp} from various places. Please refer to the
  information here: 
\url {http://canvas.gnome.org:65348/pybliographer/download.html}. 

  
  \textbf{Note:} \Pyb\ is included in all major Linux distributions.
  In most cases, you will prefer to use the specially prepared package
  that your distribution offers. See \UG \ref{UG.sec:rginstall} on
  information how to obtain and install \Pyb\ using the facilities of
  your distribution.


\item \textbf{CVS access}

  You can access the current stable and unstable code on \textbf{CVS}
  at \url{cvs.gnome.org/cvs/gnome/pybliographer}.

  For information on joining \Pyb\ development address yourself to the
  mailing list.
 
  
\end{itemize}


%%% Local Variables: 
%%% mode: latex
%%% TeX-master: "DG"
%%% End: 
