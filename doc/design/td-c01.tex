\setcounter{secnumdepth}{3}
\cleardoublepage


\section{Introduction and Generalities}
\label{sec:intro}

\subsection{The Way of Pybliographer}
\label{sec:way-pybliographer}

\paragraph{Highlights of Pybliographer Release 2}
\label{sec:highr2}
\begin{itemize}
\item The Index(-view) has been reworked, it now consumes less
  resources, starts and redraws much faster.
\item The Import facilities have been completely rewritten with a
  view to providing a \textit{import framework}, thus much reducing
  the expense involved in writing additional input filters, but also
  vastly (?) improving, at the same time, the level of support and
  comfort for every user. 
\end{itemize}



\subsection{An overview of user requirements}
\label{sec:requ}



As a bibliographic database manager, \textit{Pybliographer} places
itself at the intersection of various expectations, traditions, and
requirements, each of which developed originally independent, and
often in ignorance of the other, and which are still shaping the field
according to their own peculiarities, although they are getting into
closer contact recently.

It should prove useful to start with a view of these ancestral
developments:

In 1876 Charles A.~\textsc{Cutter} thus explained the aim of the
catalogue:\footnote{as quoted in \cite[p.196]{Hal98}}
\begin{itemize}
\item To enable a person to find a book of which either
  \begin{itemize}
  \item the author is known.
  \item the title is known.
  \item the subject is known.
  \end{itemize}
\item To show what the library has 
  \begin{itemize}
  \item by a given author.
  \item in a given subject.
  \item in a given kind of literature.
  \end{itemize}
\item To assist in a choice of a book 
  \begin{itemize}
  \item as to its edition (bibliographically).
  \item as to its character (literary or topical).
  \end{itemize}
\end{itemize}

These are still important principles, today perhaps even more than
most of the time. 




%In the following some of these strands are more fully discussed.


\subsection[The legacy of the card catalogue -- MARC]
{The legacy of the card catalogue --
  \textsf{MARC}\protect\footnote{Although \textsf{MARC} stands for Machine
  Readable Cataloguing, it is used here loosely but legitimately and
  stands for the whole of classical librarian's cataloguing, of which
  the actual MARC standard has been, in a way the epitome and
  culmination of centuries of work. Of course, it has begun to change \dots }}
\label{sec:the-legacy-card}

Originally inventories for the use of the librarians only, since about
one century, the library catalogues have become accessible to the
public and henceforth set the standard of bibliographical description.

At about the same time began the development of rules for the
bibliographical description,\footnote{The \textit{Preu�ische
    Instruktionen} and the \textit{ALA Rules} were among the first.
  Their successors are the \textit{RAK} and the AACR2, both are in
  turn influenced (the RAK more so) by the international
  standardisation efforts of the \textit{ISBD}.} to enable the sharing
of catalogue entries between libraries\footnote{Either by one library
  providing the catalogue cards f or the majority of the libraries (in
  the US), or by building union catalogues for the purpose of sharing
  the holdings, too (the inter-library lending service in Germany).}
and to enable a catalogue to be build and used \textit{compatibly}
without annoyance by many people (which is a form of sharing, too).

The actual rules used have always been a compromise between precision
and cost, oscillating between the ideal of capturing the whole title
page (of course, only the title page\footnote{see Robert Musil's
  \textit{Mann ohne Eigenschaften} and the talk between General Stumm
  von Borgwehr and the director of the Royal Library \dots}) and the
need of coping (manually) with all this. Over time, the descriptions
tended to grow in size. There are two reasons for this: 
\begin{enumerate}
\item the more titles exist, the more it becomes difficult to
  distinguish between similar ones,
\item more information is made centrally available, mostly to share the
  cost of providing it. Such was the case in Germany, where subject
  headings appeared in the national bibliographical database, thus
  moving something which has traditionally been the responsibility of
  the individual institution onwards to the \textit{Deutsche
    Bibliothek}. To do so, additional categories needed to be defined,
  rules set up, an authority file maintained, as it was done earlier
  for the nucleus of the bibliographical description.
\end{enumerate}

It should be noted, that traditionally only books have been
catalogued, together with other physical objects, that might land in a
library (\textsf{MARC} being quite comprehensive in its scope of
materials allowed). That reflects the point of view of the librarian,
who is the custodian of these said objects, buys, lends, ranges them.
The contents are not his business.  In that his view differs from the
usual patron's view, as exemplified by the Bib\TeX\ and reference
manager software (see \ref{sec:taking-fast-track} and
\ref{sec:keep-your-powder}).  This orientation is slowly changing
under the influence of Internet resources and integration with
patron's software (section
\ref{sec:standards-evolve}).\footnote{\textsf{MARC} intended, indeed,
  to support so called \textit{analytical} cataloguing as well, if
  only as a secondary task. Even the description of archival
  resources is not, or so one thought, out of its scope. This
  reflects the facts that American libraries hold archival collections
  to a far greater degree than European ones, and that these are far
  simpler than the classical European archival fonds; but remarkably
  this proved to be a failure -- a separate standard (EAD) evolved,
  this time based on XML. Perhaps the reason was the absence of any
  form of hierarchy in \textsf{MARC} databases.}

Perhaps the most important heritage from this development is the idea
of \textit{entry points}: try to establish for everything that you
catalogue attributes that are well-defined and precise enough and make
the database search-able for them. For a card catalogue this implies
defining not only under which \textit{form} an author is filed, but in
addition under which author a book is filed (if there is a
choice). The first class of normalisations is done via
\textit{authority files} the importance of which, under the conditions
of on-line catalogues, has only increased, because it is more
difficult to find entries which differ slightly \cite{Hal98}
while the so called question of the main entry has no longer the
importance it used to have. 
 


\subsection{Taking the fast track  -- BibTeX}
\label{sec:taking-fast-track}
 

Very early the scholarly (and technical) document preparation has been
supported with programs to augment document sources with correctly
formatted references (\textit{Refer} and \textit{BibTeX} are two
similar programs for the major document processing environments).

These are, in a way, the opposite of the \textsf{MARC} applications: 
the categories are more often than not only loosely defined, they are
slanted towards the production of reference lists, almost at the
exclusion of other uses, the formats are implicitly defined and
easy to extend, and the defining program is restricted to the
production of reference lists -- the building and maintaining of its own
data base, e.g., is left to other means.

In spite of (or perhaps because of) these deficiencies these have been
very successful programs. Their restrictions, that follow a well
proved tradition after all, were intended and adequate at the time of
their creation. They have been addressed by the next class of
applications which developed, characteristically, in the highly
competitive marketplace of the \textit{personal computer}, as opposed
to the more traditional scientific Unix\slash workstation environment,
that \TeX\ and Bib\TeX\ (and earlier troff and Refer) thrived in.



\subsection{Enters the Personal Computer -- Reference Managers}
\label{sec:enters-pers-comp}

With the advent of the Personal Computer a new class of users came to
the fore. While the majority of the scientific\slash technical writers
(who had little choice, anyway) stayed with their \TeX\ applications,
the new users were mostly attracted to the perceived simplicity of the
word processors, and eschewed the command line and simple editors of
Unix. Almost by necessity, the new applications stressed those aspects
of the job, that the older tradition neglected; word processor users
didn't care for the typographical refinements of \TeX, nor the
frugality and intellectual self restriction of the command line. They
wanted simplicity and ease of use, \dots\ and over the time, they
succeeded. The graphical user interface comes no longer as an
afterthought, it is at the centre of the development effort (often to
the detriment of the application proper).

The features of a commercial product are given in appendix
\ref{sec:commfeat}. 

For us it is important to pay attention to the \textit{organising} of
the references, by providing suitable instruments, e.g., the
\textit{virtual folders} of Biblioscape.


\subsection{The spell of the Internet -- Database \& Document Access}
\label{sec:internet-access}

Although the Internet precedes the personal Computer by some
years,\footnote{Of course, it all depends.  Arguably the first personal
  computer was the \textsf{Alto} at \textit{Xerox Parc} from
  the early 1970,  while the switch to the \textit{IP/TCP} protocol
  family did not occur but a couple of years later.} 
the impact of the Internet was but slowly felt. We can distinguish
the following uses:
\begin{description}
\item[Database access] 
\item[Document access] 
\item[Electronic documents] posed new challenges to the librarian's
  profession. Ephemeral, they were often felt not to warrant the
  expense of traditional cataloguing (out of which considerations the
  \textit{Dublin Core} standard evolved), hierarchical, they
  stretched the card-bound structure of \textsf{MARC} to the limits,
  ever changing, they challenged the concepts of bibliographical unit,
  of work, and edition (see \ref{sec:bibdata} and
  \ref{sec:standards-evolve}). 
\end{description}

\subsection{The view from the fringes -- exotic becomes normal}



\subsection{The standards evolve}
\label{sec:standards-evolve}
\textit{The major development has been not the Dublin Core but the
  development of an object model that underlies the cataloguing process.}





\subsection{Other lists of requirements}
\label{sec:otherreq}



The following is a list of user requirements as given in a talk in the
technical university of Chemnitz\footnote{\url{%
    http://archiv.tu-chemnitz.de/pub/2001/0013/data/anforderungen.htm}} 
\begin{itemize}

\item Flexible access via various categories, viz. author, title,
  publication year, journal, and combinations thereof.

\item Import and export in various formats, viz. BibTeX, Medline,
  Refer, RFC1807, [MARC, MAB, ...]

\item Supports working groups, allows combinations of databases.

\item Allows to merge data from various sources, and to normalise it
  (e.g., with respect to differing ways to abbreviate a journal title ...)

\item Adapts to the various formatting requirements of different
  journals, easy development of new styles.

\item Extensible by local fields, to hold notes, order information,
  annotations, excerpts, summaries, etc.

\item Simple use during document preparation: allows to search the
  references from the editor/word processor (by author, title, e.g.),
  thus eschews the use of labels -- the latter are to be used and
  provided only internally. \seet{wpintegration}

\item Allow confidential data in shared databases (e.g., annotations
  of theses).

\item N.B.: These requirements transcend BibTeX's abilities; thus it
  should be used only as an export format.
\end{itemize}

Other points that were mentioned on the mailing list:
\begin{itemize}
\item Pybliographer starts too slow; with larger databases one has no
  indication of progress, and must wait for a long time. 
\item Pybliographer holds the whole database in memory, and uses a lot
  of it! (For comparison: Allegro claims about 600 B (of secondary
  storage) per entry.)

\end{itemize}



\subsubsection{The opportunity for a flexible bibliographic format.}

The following excerpts from a recent googling\cite{mch96} I'd like to
add: The opportunity for a flexible bibliographic format -- Designing
a bibliographic format for highly varied levels of effort:

  \begin{quote}
    It should be possible to create a bibliographic format which
    permits people to use exactly the level of detail they wish,
    rather than requiring or preventing less or more.  This would
    reduce the current undesirable proliferation in bibliographic
    formats.
  \end{quote}


\paragraph{The problem:}
People create bibliographic records with enormously varying levels of
effort.  Further, the distribution of effort across different aspects
of bibliographic information also varies.  Current bibliographic
formats tend to be useful only over a relatively narrow range.  They
establish minimums by requirements on information and its form.  They
establish maximums by limitations in the expressiveness of the format.
They do not gracefully handle attempts to get by with less work, or to
add more work to further refine the record.  This is approximated by
optional fields, but the format and meaning of a fields content is
usually inflexible.  Thus there is an undesirable proliferation of
formats, a new one created whenever someone wants a new tradeoff in
ease versus precision.  The new format is generally as inflexible as those
which preceded it, and the problem thus continues. [\dots]



\paragraph{The opportunity:}
Create a bibliographic format which can gracefully handle a wide range
of cataloguing intensity.  One which could handle the complete spectrum
from Bib/Refer to MARC.  And do it in a way which permits work to be
spent only and exactly where one wishes.  Such a format would be a
`lingua franca', as any other bibliographic format could be converted
into it without losing, or requiring additional, information.[\dots]


\newpage
\subsection{Summary of Requirements}
\label{sec:summrequ}


\begin{itemize}
\item The oldest stems from the librarian's duty to keep inventories
  of his collection in form of \textit{catalogues} -- and the
  associated application software.
\item Very early the scholarly (and technical) document preparation
  has been supported with programs to augment document sources with
  correctly formatted references (\textit{Refer} and \textit{BibTeX}
  are two similar programs for the major document processing
  environments).
\item These programs concerning themselves, however, only with this
  last step, other tools developed over time to help with organising
  and searching the databases these programs require. Such
  \textit{reference managers} also aided in the odious task of
  actually creating (i.e., typing) the needed entries.
\item With the growth of Internet usage, it became possible and more
  or less usual to access databases of bibliographical interest
  directly. Both the nature of the data (format, extent, correctness,
  completeness) and of the means of access (protocols, navigation,
  access restrictions, formatting)  vary enormously. Nevertheless, it
  has become desirable to easily integrate the results of these
  searches into one database -- adding a new complexity to the tasks.
\item In addition, more and more the exchange of documents needs to be
  integrated, too. 
\item The World Wide Web may also serve as an imperfect but easily usable
  and ubiquitous way of providing \textit{access} to a database.
\item Finally, the traditional library catalogues have also added
  support for additional administrative functions (both on the
  acquisitions as on the lending side), many of which are quite
  natural extensions of their primary function.
\end{itemize}



%%% Local Variables: 
%%% mode: latex
%%% TeX-master: "td-td1"
%%% End: 
