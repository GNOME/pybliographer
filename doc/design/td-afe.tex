% $Id$

\chapter{Feature lists}
\label{cha:commfeat}


\section{User Requirements}
\label{sec:userrequire}
\newcommand{\tabcaption}{}
\newcommand{\tablabel}{}
\newenvironment{featlist}[2]{\small%
  \renewcommand{\tabcaption}{#1}\renewcommand{\tablabel}{#2}
  \begin{table}[ht]
    \centering
    \begin{tabular}[t]{|c|p{.5\textwidth}|p{2cm}|p{2cm}|}
      \firsthline Id& Description& Related& See also\\ \hline}
    {\hline   \end{tabular}   \caption{\tabcaption}   \label{\tablabel}
  \end{table}}


 The following is a list of user requirements as given in a talk in the
 technical university of Chemnitz\footnote{\url{%
     http://archiv.tu-chemnitz.de/pub/2001/0013/data/anforderungen.htm}
   Note given there: These requirements transcend BibTeX's abilities;
   thus it should be used only as an export format.}


 \begin{featlist}{User Requirements (Chemnitz)}{tab:usreqchem}
   
9001&  Flexible access via various categories, viz. author, title,
   publication year, journal, and combinations thereof.&&\\

9002&  Import and export in various formats, viz. BibTeX, Medline,
   Refer, RFC1807, [MARC, MAB, ...]&&\\

9003&  Supports working groups, allows combinations of databases.&&\\

9004&  Allows to merge data from various sources, and to normalise it
   (e.g., with respect to differing ways to abbreviate a journal title
   ...) &&\\

9005&  Adapts to the various formatting requirements of different
   journals, easy development of new styles.&&\\

9006&  Extensible by local fields, to hold notes, order information,
   annotations, excerpts, summaries, etc.&&\\

9007& Simple use during document preparation: allows to search the
   references from the editor/word processor (by author, title, e.g.),
   thus eschews the use of labels -- the latter are to be used and
   provided only internally. \seet{wpintegration} &&\\

9008& Allow confidential data in shared databases (e.g., annotations
   of theses).&&\\
 \end{featlist}


  Other points that were mentioned on the mailing lists:

\begin{featlist}{Pybliographer Deficiencies}{tab:usreqpydefic}

9010& Pybliographer starts too slow; with larger databases one has no
indication of progress, and must wait for a long time.&&\\

9011& Pybliographer holds the whole database in memory, and uses
reportedly 70MB for a medium sized file.&&\\

9012& RIS and INSPEC import formats are missing&&\\

\end{featlist}


\section{From Software Reviews}
\label{sec:reqsoftrev}

A short table is found in the website ``Bibliographic Database
Managers'', prepared by Robert Sathrum, Natural Resources Librarian of
HSU Library\cite{sat:hsu01}:

\begin{featlist}
  {Database Structure Features}{tab:hsu01db}

9100& Accommodate unlimited number of records&&\\
9101& Create more than one database. Records in one database can be
      copied or moved to other databases.  &&\\ 
9102& Numerous input forms for different citation formats, e.g.,
      journals, theses  &&\\ 
9103& Customizable field displays &&\\
9104& Sort records by different fields, e.g., date, author, journal title&&\\
9105& Sort at more than one level, e.g., first by author, then by
      journal, then by date &&\\  
\end{featlist}

\begin{featlist}
  {Database building capabilities}{tab:hsu01cr}
9106& Import records downloaded from external electronic indexes and
      catalogs&&\\   
9107& Search Z39.50 compliant databases and automatically import records&&\\  
9108& Import records from word processor files &&\\
9109& Import records manually &&\\
9110& Detection of duplicate records &&\\
9111& Spell check records as they are input &&\\
9112& Edit records individually or globally &&\\
9113& Create authority lists for selected fields, e.g., authors,
      journals, subjects  &&\\ 
9114& Define synonyms (cross-references) for related keywords &&\\
\end{featlist}

\begin{featlist}
  {Database search and retrieval capabilities}{tab:hsu01sr}
9115& Define specific fields to search &&\\
9116& Search using keywords in authority lists, e.g., author, subject&&\\  
9117& Use of Boolean operators and word truncation &&\\
9118& Search results retrieved as a separate file &&\\
9119& Ability to mark individual records &&\\
\end{featlist}

\begin{featlist}
  {Database applications}{tab:hsu01ap}
9120& Format references in multiple bibliographic styles, e.g., MLA,
      APA, CBE, to meet the requirements of scholarly  publications. &&\\ 
9121& Create independent bibliographies organized by subject  &&\\
9122& Create a manuscript and associated bibliography in one unified
      operation using a word processor  &&\\ 
9123& Export references in different file formats for use in other
      programs&&\\
9124& Export bibliographies to the web &&\\
\end{featlist}


%Quelle:                   BIBLIOGRAPHIC DATABASE MANAGERS \citep{bib:uwa00}

\clearpage
Francesco Dell'Orso \cite{ors:bfs02} provides a long evaluation scheme
from which we take the following \textit{3.3 Summary of Available  Functions}:
\begin{enumerate*}
%& 3.3 Available Functions: summary
\item Search 
\item Remote search (Z39.50) 
\item Print 
\item Export 
\item Sort 
\item Input/Cataloguing 
\item Global corrections 
\item Import 
\item Reformatting during import 
\item Input via catching www pages 
\item Manuscript formatting (also from within the wp) 
\item Managing term lists 
\item Thesaurus 
\item Duplicates detection 
\item Circulation (Loans) 
\item Can activate external files and applications (OLE or OS' Shell) 
\item Compute 
\item Graphic files management
\end{enumerate*}


\begin{featlist}
  {Dell' Orso: Generalities}{tab:ors02gen}

9125& 2 Installation and start: 1.Guided installation 2. Uninstall&&\\ 
9126& 2.1 Password required for 1. access 2. specific tasks 3. files
      4. fields&&\\ 

9127& 15 Documentation
                 1. Reference manual 
                 2. Tutorial 
                 3. Context-sensitive Help screens 
                 4. Error messages &&\\
9128& 8 Thesaurus&&\\
9129& 3.9 Macro&&\\
9130& 3.10 Mouse 1. right click 2. drag-and-drop&&\\


\end{featlist}


\begin{featlist}
{Dell' Orso: Technology Substrate}{tab:bfs02tec}
9131& 3.1 Network version
                 1. Multi-user network version 
                 2. All functions available 
                 3. Some functions can be protected (e.g. search,
                 input, output available to many, but
                 not management: rebuild, import ...) 
                 4. One write access vs. many read accesses (read =
                 search, print, manuscript  formatting) 
                 5. Many write accesses and many read 
                 6. Read only version&&\\

9132& 3.2 Internet
                 1. can use different browsers 
                 2. can automatically import/capture of WWW page 
                 3. launch URL from within a record 
                 4. can print HTML output 
                 5. client Z39.50 
                 6. can post database to the WWW that can be
                 dynamically searched&&\\


9134& 3.4 Internal database management
                 1. Create 
                 2. Rename 
                 3. Copy 
                 4. Delete 
                 5. Undelete 
                 6. Rebuild 
                 7. Save 
                 8. Compress (release space) 
                 9. Record undelete 
                 10. Revert global corrections
                 11. Statistics&&\\

9135& 3.5 Functions across different databases
                 1. search 
                 2. print 
                 3. global corrections 
                 4. duplicate detection 
                 5. manuscript formatting&&\\

9136& 3.6 Database subsets
                 1. retrieved records 
                 2. highlighted records 
                 3. marked records 
                 4. virtual sets
                 5. imported set
                 6. duplicate set&&\\

9137& 3.7 Other library functions (e.g. acquisitions, periodicals control,
                 circulation,     statistics)  
                 1. built-in 2. can integrate&&\\

9138& 3.8 Data type
                 1. textual (also ANSI 128-255) 2. numeric 3. graphic
                 4. sound 5. bar codes &&\\
  
\end{featlist}

\begin{featlist}
  {Dell' Orso: Database and record structure}{bfs02rec}
9140& 4.1 Files [used, number etc.]&&\\
9141& 4.2 Internal database and record structure&&\\
9142& 4.3 Horizontal links: between databases, records, list
      entries&&\\
9143& 4.4 Hierarchical links (es. thesaurus, mother/sons records,
  text/notecards) &&\\
9144& 4.5 Ready, predefined record structure &&\\
9145& 4.6 Input worksheets
                 1. features 
                 2. can be modified 
                 3. can create others&&\\
9146& 4.7 Fields attributes can be changed; can be applied to other
      fields &&\\ 
9147& 4.8 Multiples (multi-value) fields
                 1. present/absent 2. specific input rules&&\\
9148& 4.9 Indexed fields for searching&&\\
9149& 4.10 Record number
                 1. system assigned 
                 2. reserved, cannot be altered 
                 3. user assigned 
                 4. allows duplicate numbers 
                 5. renumbering 
                 6. is a sortable field 
                 7. is a searchable field 
                 8. can be displayed in printed output 
                 9. reuse deleted numbers
                 10. can be alphanumeric&&\\

\end{featlist}

\begin{featlist}
  {Dell' Orso: Input/Edit}{tab:bfs02inp}
9150& 5.1 Cataloguing Reference Standard &&\\
9151& 5.2 How to recall records for editing
                  1. list browsing 
                  2. query search 
                  3. while editing another record&&\\
9152& 5.3 Input must be preceeded by searching&&\\
9153& 5.4 Compulsory input within certain fields&&\\
9154&5.5 Particular edit features
                  1. copy fields and/or records 
                  2. symbols table 
                  3. term lists 
                  4. date stamping 
                  5. undo 
                  6. default values (session and/or permanent) 
                  7. validation
                  8. automatic assignment of values
                  9. configure edit window
                  10. other&&\\
9155& 5.6 Spellchecker&&\\
9156& 5.7 Direct print of a record while editing it&&\\
9157& 5.8 Duplicate record
                  1. drag-and-drop 
                  2. copy command 
                  3. clipboard (formatted) 
                  4. export/print on disk (also formatted)&&\\
9158& 5.9 Duplicates detection
                  1. fixed criterium 
                  2. criterium can be defined&&\\
9159& 5.10 Editor
                  1. upper/lowercase conversion 
                  2. cut/paste text 
                  3. b/e record 
                  4. b/e field 
                  5. delete to End-Of-Field&&\\
9160& 5.11 Font editing: font, size, bold, italics, underline,
  super/subscript, small caps &&\\
9161& 5.12 Global corrections
                  1. add string (b/e) 
                  2. locate/replace (words, case) 
                  3. delete entire field content 
                  4. move field content 
                  5. change record type
                  6. can use wildcard&&\\
\end{featlist}

\begin{featlist}
  {Dell' Orso: Import}{tab:bfs02imp}
9162& 6.1 Different ways. 
                 1. direct copy (write to database) 
                 2. batch import a) with reformatting filters a1)
                 ready-made a2) un/modifiable a3) 
                 user can define more
                 3. can capture WWW pages
                 4. can download data ready formatted in its own
                 proprietary format from specific data sources&&\\
9163& 6.2 Delimited/Tabbed stuctured input ASCII text file
                 1. fixed/variable number of fields 
                 2. fixed/variable fields position 
                 3. RT can be changed 
                 4. multiple value fields allowed 
                 5. fields also on different lines
                 6. can define field separator
                 7. can replace delimiter if embedded in field
                 8. can define end of record&&\\
9164& 6.3 Can import alpha/numeric data from a spreadsheet&&\\
9165& 6.4 Can import proprietary format files&&\\
& 6.5 Can import ISO 2709 format. &&\\
9166& 6.6 Can import MARC format file. &&\\
9167& 6.7 Tagged stuctured input ASCII text file File. Reformatting:
                 1. condition check
                 2. change RT
                 3. merge fields
                 4. delete/discard fields
                 5. field content parsing
                 6. add field content, strings
                 7. tolerate fields in variable position
                 8. upper/lower case conversion
                 9. replace text&&\\
9168& 6.8 Operability
                 1. read range of records rather than all
                 2. read one record at the time, confirm y/n
                 3. handle duplicates
                 4. preview
                 5. log file&&\\
\end{featlist}

\begin{featlist}
  {Dell' Orso: Search}{bfs02sea}
9170& 7.1 Different levels and approaches:
                 1. easy / expert 
                 2. menu/command driven 
                 3. browsing term lists / indexes 
                 4. query expressions 
                 5. browsings record list&&\\
9171& 7.2 Browsing the Search Index
                 1. entries show number of related docs
                 2. relationships (e.g. x-refs) between entries are
                 displayed  
                 3. direct selection of index terms and display of
                 related documents &&\\
9172& 7.3 Query expressions&&\\
9173& 7.4 Natural language queries&&\\
9174& 7.5 Search strategy
                 1. can save and recall search expressions
                 2. can recall previous queries within the same
                 session 
                 3. can combine previous search steps&&\\
9175& 7.6 Can save and recall search result&&\\
9176& 7.7 Shows hits of each search expression component&&\\
9177& 7.8 Can print directly one or more records while in search mode&&\\
9178& 7.9 Refine&&\\
9179& 7.10 Advanced search features
                 1. best match, weighted terms, ranking 
                 2. fuzzy, sounds like 
                 3. hypertext-like&&\\
9180& 7.11 Response time&&\\
9181& 7.12 Instant display of retrieved records
                 1. short record list 
                 2. one record at the time&&\\
9182& 7.13 Highlighting search terms in result ( + jump to next occurrence
      of term)&&\\
9183& 7.14 Indexing operation
                 1. automatic, real time 
                 2. batch&&\\
9184& 7.15 Indexing techniques
                 1. any character string 
                 2. word by word 
                 3. phrase (adjacent words) 
                 4. marked portions of fields&&\\
\end{featlist}
\begin{featlist}
{Dell' Orso: Search \textit{continued}}{tab:bfs02seb}
9185& 7.16 Scope of searching
                 1. one or more distinct fields
                 2. cluster of fields 
                 3. full text = any field
                 4. same occurrence&&\\
9186& 7.17 Case sensitiveness&&\\
9187& 7.18 Diacritics&&\\
9188& 7.19 Can use and nest parenthesis&&\\
9189& 7.20 Priority within search operators and queries&&\\
9190& 7.21 Boolean operators. 
                 1. AND 
                 2. OR 
                 3. NOT (unary) 
                 4. AND NOT (binary) 
                 5. XOR&&\\
9191& 7.22 Relational operators
                 contains, <>, <, <=, >, >=, range, equal&&\\
9192& 7.23 Can combine boolean and relational operators&&\\
9193& 7.24 Truncated search
                 1. explicit or implicit 
                 2. right 
                 3. left 
                 4. r/l&&\\
9194& 7.25 Can search by position: b/e field and/or occurrence&&\\
9195& 7.26 Search for not/empty fields&&\\
9196& 7.27 Internal wildcards (e.g. *?)&&\\
9197& 7.28 Can combine boolean, relational, parenthesis, truncation etc.&&\\
9198& 7.29 Adjacency and proximity operator&&\\
9199& 7.30 Search only within the same occurrence of a repeatable field&&\\
9200& 7.31 Search only within the same paragraph&&\\
9201& 7.32 Stopwords&&\\
9202& 7.33 Input allowed while indexing&&\\
9203& 7.34 Z39.50 Searching&&\\
\end{featlist}



\begin{featlist}
  {Dell' Orso: Output and Print}{tab:bfs02out}
9294& 9.1 Send output to printer, file, video&&\\
9295& 9.2 Real bibliography formatting software: offers large quantity of --
      ready and modifiable -- styles, can create new&&\\ 
9296& 9.3 Structure of system display
                 1. short record list 
                 2. one formatted record at the time
                 3. more formatted records&&\\
9207& 9.4 Report generator&&\\
9208& 9.5 "Subject list", i.e. list with sorted headings from record content
                 1. one level 
                 2. more levels
                 3. headings only&&\\
9209& 9.6 Output file format 
                 1. RTF 2. Word 3. WP 4. TXT 5. HTML 6. other. &&\\
\end{featlist}




\begin{featlist}
  {Dell' Orso: Formatting language to define output styles}{tab:bfs02fla}
9210& 10.1 FL Selection 
                  1. fields 2. subfields. &&\\
9211& 10.2 FL Can add text
                  1. in front of/after 
                  2. regardless of field presence 
                  3. depending on field presence 
                  4. depending on field content&&\\
9212& 10.3 FL Can distinguish among occurrences of a repeatable field:
                  1. by punctuation -separators 
                  2. because of position / sequence number 
                  3. can count them&&\\
9213& 10.4 FL Can produce tagged format output (e.g. to export)&&\\
9214& 10.5 FL can display RT&&\\
9215& 10.6 FL can produce permuted indexes (words in-out-and
      context)&&\\
9216& 10.7 FL offers conditional commands (IF ... THEN...)&&\\
9217& 10.8 Upper/lowercase conversion&&\\
9218& 10.9 Look-up tables to expand acronyms, abbreviations, replace
      text&&\\ 
9219& 10.10 Contextual Record Preview&&\\
9220& 10.11 Text added in styles can be language dependent for each record
                  1. text lists can be modified
                  2. new lists can be added (new language)
                  3. text can be present in various fields&&\\
9221& 10.12 Check format syntax&&\\
9222& 10.13 FL Level of difficulty&&\\
\end{featlist}

\begin{featlist}
  {Dell' Orso: Sort}{tab:bfs02srt}
9223& 11.1 Scope
                 1. sorting records within the database 
                 2. sorting records in output &&\\

9224& 11.2 Basic sort criterium for characters:
                 1. MS-Windows tables 
                 2. program specific settings and/or tables
                 3. user defined table (es. ?= ae)&&\\

9225& 11.3 Sort keys on different levels&&\\

9226& 11.4 How to sort the database&&\\

9227& 11.5 How to sort for printing&&\\

9228& 11.6 Database sort is kept over sessions&&\\
9229& 11.7 Sort key length can be defined&&\\

9230& 11.8 Sort for printing belongs to output styles&&\\

9231& 11.9 Sort records in output using: 
                 1. words 
                 2. single occurrence of a multiple field 
                 3. marked strings 
                 4. portions of field and subfield 
                 5. whole field&&\\

9232& 11.10 Edit can alter the sort value of a string&&\\

9233& 11.11 Conditional commands available for sorting&&\\

9234& 11.12 Sort keys derived from different fields&&\\

9235& 11.14 Sorting speed&&\\

9236& 11.13 Ignore initial articles and punctuation marks&&\\

9237& 11.15 Other&&\\
\end{featlist}
\begin{featlist}
  {Dell' Orso: Sort \textit{continued}}{tab:bfs02sr2}
9238& 11.16 Sort can produce headings above sorted records ("subject
   bibliography")&&\\

9239& 11.16 Sort can produce headings above sorted records ("subject
   bibliography")&&\\

9240& 11.16.1 Sort keys are another item from headings, thus
                can match or be different &&\\        
9241& 11.16.2 Headings might not be displayed within records&&\\
9242& 11.16.3 More than one level of sort key as headings&&\\
9243& 11.16.4 Sort of records under the same sort key&&\\

9244& 11.16.5 Sort occurrences of a repeatable field as headings
                 1. altogether 2. all separated 3. just one&&\\

9245& 11.16.6 Records referenced more than once by different sort headings&&\\


9246& 11.16.7 Sort headings can be formatted&&\\

9247& 11.16.8 Can produce indexes referencing records in the database by a
                 short element            (e.g. RN)&&\\

\end{featlist}

\begin{featlist}
  {Dell' Orso: Export}{tab:bfs02exp}
9448& 12.1 Export formats
                 1. delimited: comma, tab, <CR>, other 
                 2. tagged
                 3. ISO 2709 
                 4. MARC 
                 5. proprietary format of other database&&\\

9249& 12.2 Fields that can be exported
                 1. all 
                 2. some 
                 3. RT 
                 4. RN&&\\
\end{featlist}


\begin{featlist}
  {Dell' Orso: Manuscript formatting}{tab:bfs02for}

9250& 13.1 Compatible wordprocessors&&\\

9251& 13.2 On-line contextual help&&\\

9252& 13.3 Can format more than on document at the time&&\\

9253& 13.4 Can generate bibliography from more than one database at the time&&\\

9254& 13.5 Entering placeholders within text
                 1 from within wp text (ad hoc tool bar and/or pull
                 down integrated menu): 1a 
                 manually writing; 1b automatic insert
                 2 from within db: 2a ad hoc command (operational
                 in-text placeholder); 2b via 
                 clipboard (in-text format has to comply with
                 placeholder format) &&\\

9255& 13.6 Location for placeholders
                 1. main text 2. end/footnote 3. hidden text&&\\

9256& 13.7 Content and structure of placeholders
                 1. author name (1a any order)
                 2. title
                 3. keywords
                 4. date
                 5. RN 
                 6. one or more, also truncated, string from any field
                 7. delimiters/markers can be changed 
                 8. within the same style as many as RT&&\\

9257& 13.8 Different references same author same year
                 1. must differentiate within db (1990b) 
                 2. program can distinguish them
                 3. must differentiate within text&&\\

9258& 13.9 Multiple citation (same or different authors)
                 1. can sort
                 2. can join: 2-5
                 3. can suppress repeated names
                 4. can insert text&&\\
9259& 13.10 Final in-text citation format is different from placeholder&&\\

9260& 13.11 Final in-text citation format is ruled by ad hoc style within db
      made up of:  
                 1. citation number
                 2. author-date
                 3. other
                 4. shortened version of complete bibliographic reference
                 5. two styles: in-text vs. note
                 6. nothing&&\\
\end{featlist}


\begin{featlist}
  {Dell' Orso: Manuscript formatting}{tab:bfs02fo2}

9261& 13.12 Changes to in-text citation standard format
                 1. add text (in front of / after);
                 2. hide portion;
                 3. hide completely;
                 4. put only in the list, exclude in-text;
                 5. first different from its repetitions &&\\

9262& 13.13 Bibliographic reference list format ruled by db style different
      for each RT 
                 1. generated within wp text
                 2. generated in a copy of wp text
                 3. settings remain for next session
                 4. can automatically exclude citations placed in notes
                 5. list can include references not shown as in-text
                 citations&&\\

9263& 13.14 Word processor or db control reference list format
                 1. Paper size, margins, headers footers;
                 2. Heading;
                 3. Font, size;
                 4. Indent, line  spacing;
                 5. Citation numbering, pre/suffix, justification&&\\

9264& 13.15 Duplicates are automatically removed from the list&&\\

9265& 13.16 Reference list sort order
                 1. citation order;
                 2. sort order defined within db A/D;
                 3. sort order defined within wp by the program&&\\

9266& 13.17 In-text citations and list references can be changed without
      changing wp text &&\\
9267& 13.18 Errors
                 1. during inserting and formatting; 2. user can act
                 on the spot; 3. log file &&\\

9268& 13.19 Manuscript formatting takes place:
                 1. within wp text (1a same document 1b other document)
                 2. within db
                 3. in one step
                 4. in more than one step&&\\

9269& 13.20 More citations of the same reference can bear same number&&\\
\end{featlist}

\cbstart

\begin{featlist}
  {Dell' Orso: Term/Entry list, authority file}{bfs02aut}

9270& 14.1 Fixed number& 3.0.01&\\

9271& 14.2 Lists' content is automatically derived from db data (or can
     contain external data) & 3.0.03&\\

9272& 14.3 Lists are physically separated from database&&\\

9273& 14.4 List reflects records content in real time& 3.0.03&\\

9274& 14.5 List can be directly edited&&\\

9275& 14.6 When list entries are edited, records change&&\\

9276& 14.7 New entries are validated (go list: new, old, probably a
     duplicate) &3.0.06&\\

9277& 14.8 List entry can contain its own supplementary data: note,
     abbreviation, date,    compiler, x-refs &&\\

9278& 14.9 List can be printed&&\\

9279& 14.10 Import external data into the list&&\\

9280& 14.11 Lists are useful for input& 3.0.06&\\

9281& 14.12 Lists are useful for searching& 3.0.05&\\

9282& 14.13 List entries show total number of related documents& 3.0.09&\\

9283& 14.14 Lists can be shared among different db&&\\

9284& 14.15 Where and how they are used
                 1. browsing, display related records
                 2. search expressions, pick up terms from one or another
                 3. input, pick up terms from one or another
                 4. output&&\\

9285& 14.16 How lists are created and updated
                 1. input: new entries automatically update the list
                 2. ad hoc command to edit lists out of records
                 3. import
                 4. as external text file&&\\

9286& 14.17 How lists are printed
                 1. from the outside as text file 
                 2. from the inside by ad hoc printing function&&\\




\end{featlist}

\cbend

\clearpage
\section{Biblioscape}
\label{sec:biblioscapefeat}


                                                               
%\textbf{Biblioscape 4 Feature Matrix}


\textit{Organize references with folders, dynamic folders, etc.}
\begin{itemize}
 \item[Folder] \label{biblioscapefolder}
Add references into folders. One reference can be put into
 multiple folders without creating a duplicate. Organize references
 into folders with drag and drop.

 
 \item[Dynamic folder] Organize and save queries into a tree structure.
 All references meeting the search criteria will be listed under a
 dynamic folder.

 \item[Indexed search] Return search results in a couple of seconds no
 matter how big the database is. One line search works the same
 way as most Internet search engines. Supports logical searches,
 fuzzy searches, etc.

 \item[Import filter] Bibliographic data from any data sources can be
 imported with a proper import filter. User can create new or edit
 existing import filters.

 \item[Output style] References can be displayed any any style like MLA,
 APA, etc. A large number of styles are provided for different
 journals. Users can also create new ones.

 \item[Cross linking] Link a reference to other references in the same
 database. You can define a relationship for the links, like
 "Supportive", "Contradict". You can also add comments for each
 link.

 \item[Navigation view] A reference can be displayed in an organizational
 chart where each node can lead to related records.

 \item[Formatted preview] Display a reference as formatted text
 according to the active output style. 

 \item[Live preview] Display data fields of the selected reference in a grid
 without opening it. Changes made to the data will be saved to
 database as you move to another record.

 \item[Graphics and OLE] If a reference has associated graphics and OLE
 objects, add them to the Document field. The document field can
 be used to store the full text of a reference.

 \item[Field lookup] List all unique values along with the number of
 occurrences of a data field. All data fields with possible repeated
 values can be shown in lookup view. These include Author,
 Keyword, Publisher, Language, Country, Subject, etc.

 \item[Recycle bin] All deleted references are put into the Recycle bin.
 You can recover them from the Recycle bin or remove them
 permanently from the Recycle bin.

 \item[Advanced search] Query any data field with a visual query builder.

 \item[Find and Replace] Search for a word or phrase and limit the search
 to a data field or all data. The same is true for the Replace
 operation.

 \item[Sorting] Sort a column by clicking on the header. Click again and
 the column will sort in reverse order. User can also define
 multi-level sorting.

 \item[Filtering] Define a filtering criteria with a visual filter builder and
 apply the filter to any dataset.

 \item[Term list] Users can keep frequently used phrases in a term list.
 Terms can be organized in the list by category.

 \item[Move field] The content of a data field can be moved from one
 field to another.

 \item[Global edit] The content of a data field can be changed at once
 for all selected references.

 \item[Eliminate duplicates] Duplicate records can be found and
 removed. Fuzzy search is supported for finding duplicates.

 \item[Analyze references] Data fields in the reference table can be
 analyzed for data distribution.

 \item[SQL commands] Users who are familiar with SQL can query the
 database directly with SQL commands.

 \item[Report] A built-in database report writer will a print data report,
 including a subject bibliography grouped by keyword, author, year,
 subject, etc.

\end{itemize}
\textit{Format papers to generate citations and bibliography}

\begin{itemize}
\item  [Format a paper] Convert the temporary citations of a document
 into formatted citations and a bibliography.

 \item[Unformat a paper] Convert a Biblioscape formatted paper back to
 unformatted form (with temporary citations) so citations can be
 added or deleted before the final formatting.

 \item[Word support] Full integration with Microsoft Word, Biblioscape
 menus and toolbar can be added to the Word menu and toolbar
 system.

 \item[WordPerfect support] Full integration with Corel WordPerfect,
 Biblioscape menus and toolbar can be added to the WordPerfect
 menu and toolbar system.

 \item[Other word processors] Biblioscape methods for word processors
 integration are published and open to all word processors that
 support DDE.

 \item[HTML support] Biblioscape can generate formatted papers in HTML
 format. A hyperlink can be created automatically between an
 in-text citation and its reference in a bibliography.

 \item[Natural citation] Use words or phrases to uniquely identify a
 reference in a temporary citation instead of using a Reference ID.
 If references are moved into another database, temporary citations
 don't need to be changed.

 \item[Cite while you write] Use BiblioSidekick to display references in a
  small, always on top windows. While in a word processor like Word
 or WordPerfect, just drag and drop the selected reference in the
 place where you want to cite it.

 \item[BiblioWord] A full featured word processor inside Biblioscape. Just
 drag selected references from a panel on the right when you want
 to cite. BiblioWord supports live spelling check, thesaurus, tables,
 graphics, OLE, multi-level undo, etc.

\end{itemize}

\textit{Access the Internet to capture bibliographic data, Web pages}
\begin{itemize}

 \item[Remote databases] Access thousands of remote bibliographic
 databases on the Web with an integrated Web browser. These
 sites include university sites, commercial databases, and
 government sites. Most of them are free.

 \item[Capture references] Search web based bibliographic databases
 from inside Biblioscape, click a button to capture search results
 into a Biblioscape database with the right import filter. New import
 filters can be created by users.

 \item[Capture Web pages] Research on the Web with the Biblioscape
 integrated browser, capture a web page into a Biblioscape
 References table or Notes table. All words in the Web page will be
 indexed for future search. Graphics and links are captured along
 with the page.

 \item[Resources] A directory of bibliographic resources on the Web. Each
 entry listed has an associated import filter. The local Resources list
 can be expanded and edited by the user.

 \item[Web directory] Biblioscape Web site lists a collection of sites
 valuable to researchers. Web sites are organized by subject.
 Bibliographic databases are the main part of the listing. Although
 other types of Web resources are also listed.

 \item[Z39.50] Most Z39.50 enabled bibliographic databases also have a
 Web interface, Biblioscape's integrated Web browser can be used
 to search such sites and capture search results directly into a
 database. 

 \item[Link to a note] Easily create a link between a note and a Web site.

\end{itemize}


\textit{Take notes and link them to references, tasks, web sites, etc.}

\begin{itemize}
 \item[Tree structure] Organize notes in a tree structure. Note's position
 in the tree can be rearranged by drag and drop.

 \item[Indexed search] Find your note fast with indexed search. Each
 word in your Notes database is indexed for super fast search. The
 search words are colored in red on the hit page. Indexed search
 supports logical operators, wildcards, fuzzy search, etc.

 \item[Advanced search] Limit your search to a data field like Date
 Created, Keywords, etc. Build complex searches with a visual query
 builder.

 \item[Format text] The text in your note can be formatted with all the
 standard options, including fonts, color, background color,
 superscript, subscript, paragraph alignment, bullet list, number list,
 etc.

 \item[Link] Each note can be linked to other notes, references, tasks,
 catalog items, Web URLs, local files, etc. Double clicking on a link
 will take you to the linked item.

 \item[Web capture] Notes can be used to organize captured Web pages.
  All the graphics and hyperlinks of captured web pages can be
 properly displayed.

 \item[Table support] You can insert tables in your notes. Additional rows
 can be added and deleted.

 \item[Find and Replace] Standard Find and Replace tools for finding and
 replacing text in your notes.

 \item[Graphics and OLE] Graphics can be added to your notes. OLE is
 also supported. Therefore, you can add chemical structure
 drawings, spreadsheets, CAD drawings, etc. in your notes.

 \item[Table view] The notes can also be displayed in a table besides the
 default tree view. Notes can be sorted and grouped in a table.

 \item[Keyword lookup] Each note can have associated keywords. These
 keywords can be displayed in a lookup list along with its number of
 occurrences. Double clicking on a keyword will retrieve all related
 notes.

 \item[Spelling and thesaurus] A powerful spelling checker is included.
 Additional dictionaries can be downloaded for all major European
 languages. A thesaurus is also included to help the user to find the
 right words during writing.

 \item[Icons] Each note can be assigned a different icon to distinguish it
 from other notes.

 \item[Export] Each note can be exported to a file in RTF or HTML format.

\end{itemize}

\textit{Manage tasks and organize your research ToDo list}
\begin{itemize}

 \item[Sort tasks:]Click on the column header to sort tasks, click again to
 sort in reverse order.

\item[Group tasks] Group tasks by Priority, Status, Date Created, etc.
]
 \item[Task progress] Track the progress of a task by marking its
 percentage completed.

 \item[Task creation] Create tasks inside References module, and add
 selected references into the Description field of the new task.

 \item[Link to a note] Create a link between a selected task and a note.

 \item[Advanced search] Search tasks with a visual query builder.

\end{itemize}

\textit{Draw a chart to present your ideas}
\begin{itemize}

 \item[Flow chart] Draw a flow chart with an easy to use chart editor.

 \item[Knowledge map] Draw a chart and link a chart object to other
 modules. For example, double clicking on a chart object will open a
 group of references, tasks, notes, etc. A SQL query can be
 associated with each chart object. A knowledge map can be built
 with such associated queries.

 \item[Tree structure] Organize your charts in a tree structure. The
 position of each chart in the tree can be rearranged by drag and
 drop.

 \item[Link to a note] Create a link between a chart and a note.

 \item[Zoom Display] options like zoom in and zoom out, actual size, and
 fit to screen are supported.

 \item[Icon] Each chart can have an icon associated and displayed.

 \item[Shape and color] The shape and color of each chart object can be
 customized. The label text can be displayed in different fonts and
 colors.

 \item[Connectors] Chart objects can be connected with a flexible
 connector which can be curved. A connector can have its own
 label, font, color, size, different sources and destination arrows,
 and link points.
 
\end{itemize}


\textit{Manage a library without a steep learning curve}
\begin{itemize}
 \item[Catalog] Manage library collection data into 56 data fields,
 organized into several groups including Bibliographic, Holding,
 Request, Order, Serial, and General.

 \item[Serials] Manage serials and related activities including tracking,
 routing, etc. 

 \item[Circulation history] Search, sort, and group circulation data.
 Display circulation activities by borrower, status, subject, etc. 

 \item[Check Out] Check out books for library patrons, add notes, easily
 change due dates.

 \item[Check In] Check in books returned by borrowers. Automatically
 reminds librarian about Hold status.

 \item[Renew] Renew books for borrowers, add a note. Find renewed
 items by ID or title.

 \item[Hold] Put a hold on a checked out book. Show a reminder when
 that book is returned.

 \item[Interlibrary Loan] Manage interlibrary loan requests, track loan
 status, log shippings, etc.

 \item[Borrowers] Manage borrower's information (address, phone, fax,
 email, et]c.)

 \item[Lenders] Manage lender's information (contact's name, phone, fax,
 email, notes, etc.)

 \item[Suppliers] Manage supplier's information (address, phone, fax,
 email, notes, etc.)

 \item[Sort] Click on any column header to sort then click again to sort in
 reverse order.

 \item[Group] Group data by drag and drop. Data can be grouped at
 multi-levels by any data field.

 \item[Field chooser] Choose which data fields to include in the data grid
 by drag and drop.

 \item[Report and print] Build or customize data reports with a powerful
 report builder. Users can create new reports with a wizard. New
 reports can be easily added to the menu system. Reports can be
 previewed, printed, or saved as a files.


\end{itemize}

\textit{Web enable your bibliographic database with one click}

\begin{itemize}

 \item[Web publishing] Publish databases on the Web with BiblioWeb
 server. No other web server required. Runs on any Windows 95, 98,
 Me, NT4, 2000 machine.

 \item[Indexed search] Search references with a powerful search engine.
 Enter search commands like you do with a Web search engine.
 Supports search keywords AND, NOT, OR, LIKE, NEAR, Wildcards,
 etc.

 \item[Advanced search] Limit searches to certain fields. Build complex
 queries with up to 3 conditions.

 \item[Add references] Users with a Write account can add new
 references to the database using a web browser.

 \item[Edit and delete] Users can edit or delete their own references over
 the web.

 \item[Import] Import references over the Web with the right import filter,
 so you don't need to enter references one by one.

 \item[Hyperlinks] Search results are displayed with hyperlinks. Clicking
 on the hyperlink will trigger a new search for related items.

 \item[Style] Marked references can be displayed in any of the output
 styles that exist in Biblioscape.

 \item[Export] Marked references can be exported in several formats to be
 easily imported into other programs.

 \item[Format papers] Users can even format a paper over the Internet.
 Temporary citations in a document will be converted to formatted
 citations and bibliographies.

 \item[User forum] Includes a user forum application, so you can host a
 web based forum without extra cost.


\end{itemize}

%%% Local Variables: 
%%% mode: latex
%%% TeX-master: "DG"
%%% End: 
