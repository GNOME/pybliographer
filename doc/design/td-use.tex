\chapter{Organising and Manipulating References }
\label{cha:basicwork}

One aspect of Pybliographer is its use to \textit{organise references}
-- that lays the accent onto the variety of uses that these
references could be put to, and the concomitant variety of possible
organisations we have to cope with.  This de-emphasises the
bibliographical description, as a rule, but stresses our ability to
annotate and link.

As a rule, our data is kept in \textit{one} database, organised into
\textit{folders} (compare Biblioscape \ref{biblioscapefolder}).  A
variety of \textit{Indices} are kept to allow fast access.

\cbstart
\begin{dnote}

\item Most entities used in this connection are fully blown DB objcts,
  that can hold metadata, annotations, etc.  Possible exceptions are
  \textit{keywords,} 
\item Another question not yet discussed is \textit{indexing} -- but
  there are strong connections. 
\item Checking against a simple index could be used alongside spelling
  correction during input, that is what many programs do. 

\end{dnote}

\cbend

\section{Folders and Lists}
\label{sec:folders}

We often have to do with the need to structure the search space, to
group entries, and generally to organise the data in various ways.

One way to do so, which is perhaps particularly useful and of near
universal application, is to use folders. These are modelled after the
Biblioscape feature of this name, cf. also Fowler's \textit{Portfolio
  pattern}. Lists are in this connection simply stand-alone (unrooted)
folders that may contain any number of items but do not allow
sub-folders.



With each folder are associated various attributes and data structures
besides the items that it contains (or the subfolders, resp.).
\begin{itemize*}
\item Display format: in particular for special purpose lists.
\item Sorting order: same considerations. These could be combined, but
  are of unlike type.
\item Filter condition.
\item Base set.
\end{itemize*}

Major operations:
\begin{itemize*}
\item Add (move, copy) an item into one list or folder. 

Supposedly more often items are marked into a list, or subsumed into a
folder.

\item Display a folder (means display the union of all subfolders if
  not a leaf folder).
\item Open a folder (means display if a leaf folder).
\item Close a folder.
\end{itemize*}

In Biblioscape there is a so-called \textit{dynamic folder}
capability, meaning a folder the extend of which is determined by
executing a SQL [like] query against the data base. I.e., a standard
folder is defined by an arbitrary assignment, while a dynamic folder
is defined by the value of one or more attributes. It is evident that
this is an interesting variant.

To a degree, the assignment to a folder could imply the assignment of
an attribute, but it is not clear if this direction there is much to
be gained.

I would argue against tying the folders to any extrinsic purpose or
organisation, but allow them to be freely used by the user.  This does
not exclude (so it seems) using  \textit{existing} attributes for
classification, while acknowledging their limitations in most cases.


\subsection{Lists}
\label{sec:orglists}

Any result set is a \textit{List}, and vice versa. So lists are a very
prominent feature in \Pyb\ -- by using them we allow intermediate
results to be saved, as task related information to be easily kept.

List cannot be nested, to distinguish them from folders and to relieve
the application from the problem of inferring an hierarchy, but they
could be turned into folders, if the need arises.


\section{Keywords}
\label{sec:orgkeys}

It is easy to add a \textit{keyword} to an entry and easy to search
for it.  So this is an favourite feature.  But this is also a technique
that has its limitations; let us consider some situations to find out
more:

We may simply think of coarsely classifying an item, putting it onto
one heap or another, where it is understood that it can be very
diverse schemes involved, i.e., the one want to classify according to
his work plan, the other according to tasks involved, a third assigns
a priority.  This is, however, a task that is much more better
accomplished with folders, indeed, this is their foremost purpose.

Another use is to indicate the contents, as a sort of diminutive
\textit{subject headings} (see \textit{infra} section
\ref{sec:orgsubj}). Compared to these they have the following
deficits:
\begin{itemize*}
\item There is not \textit{thesaurus structure}
\item There is no support for multi-lingual application
\item In consequence, little stability can be expected in its
  application
\end{itemize*}
Therefore, it is of little interest in that respect.

However, another usage that in fact might not come easily into one's
mind is valuable: adding search terms to compensate for variations in
spelling, terminology or for omissions in titling that would otherwise
require more elaborate mechanisms.



\newpage

\cbstart

\section{Subject Headings}
\label{sec:orgsubj}

\textit{Subject headings} are, by contrast, keywords with special
enhancements which make them more useful for the purpose of indicating
the content of an item. 

Subject headings are like keywords given to an item in order to
indicate its content, but they have a \textit{syntactic structure}
that sets them off from keywords, and they are taken from a
\textit{thesaurus}, a controlled vocabulary with special features for
improved usability, both at the time the item is described and at the
time the search is formulated.

\textbf{Note:} Some interesting contributions are in the little volume
\citep{dbi:SWD90}.

An item is given one or more subject headings (up to six in Germany),
each of which is a chain of one or more (up to six in Germany) subject
elements (my translation).


\newenvironment{requtab}[3]{%
  \label{sec:rq#1}\def\myrequprefix{#2}
  \fbox{\textbf{\sffamily #2 #3}}\hrule\noindent\nopagebreak}
{\nopagebreak\vspace{\baselineskip}\hrule\medskip}
\newcommand{\requel}[5][]{%
  {\makebox[15mm]{\textbf{\sffamily \myrequprefix.#2 \hfill#1\,}}
    \parbox[t]{9.5cm}{#3 \small\textit{#4}}\,
    \parbox[t]{3cm}{\small\sffamily#5}\par}}

\begin{requtab}
  {a33}{3.3}{Subject Headings}
  
  \requel{01} {It shall be possible to add Subject Headings to
    Entries.}  {Minimally for all manifestations, but also valuable
    for e.g., quotes and notes}{}

  \requel[?]{02}
  {It shall be possible to distinguish primary and secondary references}  
  {In the sense of giving a weight.}{}

  \requel{03}
  {It shall be possible to limit the  number of subject heading per entry}
  {Some rules require this -- including zero depending upon type, e.g.}{}

  \requel{04}
  {It shall be possible to derive subject headings during import}{}{9271}

  \requel{05}
  {The assignment of subject headings (i.e. manual entry) shall be assisted}
  {by look up, browsing, completion, copying etc.}{9276, 9280}

  

  \requel{10}
  {Chained subject headings shall be used}
  {Syntactic structure v. infra}{}


  \requel{11}
  {Types of heading elements shall be distinguished}
  {persons, geographical names, historical periods, forms}{}


  \requel[?]{12}
  {It shall be possible to limit permutations of the chain}
  {(This is not without problems, though.)}{}


  \requel{13}
  {Predefined chains shall be available}
  {v. infra}{}

  \requel{14}
  {Subject headings shall be entities, not strings}
  {i.e. shared not copied -- generally so}{9273, 9275}

  \requel{15}
  {Subject headings shall be editable, im-\slash export- and
  annotatable}
  {}{9272, 9277, 9278, 9279}

  \requel{20}
  {Syntactical searching must be used}
  {v. infra}{}


  \requel{21}
  {PF references must be followed automatically}
  {}{}


  \requel{22}
  {A folding display must be used}
  {}{}


  \requel{30}
  {Multiple schemas must be supported}
  {To allow mapping during  or after import, e.g.}{}


  \requel{31}
  {A Thesaurus structure must be used}
  {}{9113 9114 9128 }


  \requel{32}
  {Multiple languages must be supported}
  {}{}

  \requel{33}
  {Import mappings must be possible}
  {}{}

  \requel{34}
  {Adding a  classification to SHs must be possible}
  {(This is done in Z�rich, e.g.)}{}

  \requel{35}
  {It should be possible to produce (hierarchically) sorted list,
  e.g., for subject bibliographies}
  {}{9208}

  
  
\end{requtab}

\subsection{Syntactic Structure}
\label{sec:orgsubsyn}

Why do subject headings need a syntactic structure?  Isn't this only
good for card catalogues?  But consider the following.

There are indeed two kinds of syntactical structure present. First we
compose subject headings as chains of subject elements in order to
make more precise the intended meaning (subordinating index).  The
improvement is, of course, more easily felt with greater databases.
But even for smaller databases the following considerations are
important:

If some subject elements occur very frequently, in variable
combinations, a worthwhile gain in precision results compared to mere
juxtaposition.  This is for example the case for papers situated on
the border between two disciplines, or on the horizon of one's
interests, where the choice of headings becomes less specific. 

It is also easier to produce and to peruse both, printed and displayed
lists and registers alike, if an additional structure is available.

And, whenever we import data we might well get data structured in this
way, why should we destroy information neddlessly?

A second kind of syntactic structure that must be taken into account,
is given within each element: the qualifying information that is used
to distinguish between homonymes. It is particular important to avoid
the qualifiers when searching, as e.g. in \textsf{group
  <mathematics>}, lest a search for \textsf{mathematics} should
produce all entries indexed with the former. 

\cbend

\subsection{Thesaurus Structure}
\label{sec:orgsubthes}



\section{Classifications}
\label{sec:orgclass}

Although a \textit{thesaurus} and a classification share a common
storage structure, they stem from different principles, are build and
used in quite different ways, and as a result complement each other
more than that they would compete.


\newpage

\cbstart

\section{Indexes}
\label{orgindex}

Indexes are, of course, a rather general concept; it is treated here,
and not in the database chapter, because they are essentially in their
unrefined way, used in many programs to emulate  what \Pyb\
accomplishes with more sophistication. 

Thus we can often find the following picture (see, e.g., the
evaluations by F. Dell' Orso \citep{ors:bfs02})

\begin{itemize*}
\item a field is indexed, sometimes every field gets its own index,
  perhaps separating authors from editors, 
\item we deal usually with strings, thus there is no option of
  annotating, or of silently following references
\item sometimes we have copy semantics,\footnote {Which is, b.t.w.,
    the rule in American library applications, such as OCLC.}
\item the index is often used  as data entry helper, checking for
  typos, or offering completion
\end{itemize*}

%%%%%%%%%%%%%%%%%%%%%%%%%%%%%%%%%%%%%%%%%%%%%%%%%%
\newenvironment{requtab}[3]{%
  \label{sec:rq#1}\def\myrequprefix{#2}
  \fbox{\textbf{\sffamily #2 #3}}\hrule\noindent\nopagebreak}
{\nopagebreak\vspace{\baselineskip}\hrule\medskip}
\newcommand{\requel}[5][]{%
  {\makebox[15mm]{\textbf{\sffamily \myrequprefix.#2 \hfill#1\,}}
    \parbox[t]{9.5cm}{#3 \small\textit{#4}}\,
    \parbox[t]{3cm}{\small\sffamily#5}\par}}

%%%%%%%%%%%%%%%%%%%%%%%%%%%%%%%%%%%%%%%%%%%%%%%%%%


\begin{requtab}{a3a0}{3.0}{General considerations -- Indexes}

  \requel{01}
  {The creation of indexes shall be freely possible}{}{9270}
  
  \requel{02} {The creation of Indexes shall be usually possible with
    little or no programming, by subclassing or cloning}
  {}{}

  \requel{03}
  {Indexes shall be maintained \textit{real-time} with option of
    backgroud (re-)building}
  {}{9271}

  \requel{04} 
  {Multiple fields (data elements) shall be able to share one index}
  {}{}

  \requel{05} 
  {All indexes shall be searchable}
  {}{9281}

  \requel{06} 
  {All indexes should be usable during input for validation or completion}
  {}{9280}

  \requel{07} 
  {Application should provide sufficient utilities for printing etc.}
  {}{9278}

  \requel[?]{08}
  {Indexes should be shareable}{}{9283}

  \requel{09}
  {The number of matching records should be displayed in appropiate
  circumstances} {}{9282}


\end{requtab}

\cbend

%%% Local Variables: 
%%% mode: latex
%%% TeX-master: "DG"
%%% End: 
